\packagename{RANDOM\_MATRIX}
\version{1.0.0}
\versiondate{6 March 2014}
\purpose{
This package generates a random sparse matrix of specified size and density in
compressed sparse column format. Either the pattern or both the pattern and
values can be generated. Both symmetric and unsymmetric matrices can be
generated, and structural non-degeneracy can optionally be ensured, and the
row indices can be sorted within columns.
}

\title{Generator of pseudo-random sparse matrices}
\author{Jonathan Hogg (STFC Rutherford Appleton Laboratory)}
\pkglang{Fortran}
\spralmaketitle
\thispagestyle{firststyle}

\versionhistory
\begin{description}
\item[2014-03-06 Version 1.0.0] Initial release
\end{description}

%%%%%%%%%%%%%%%%%%%%%% installation %%%%%%%%%%%%%%%%%%%%%%

\section{Installation}
Please see the SPRAL install documentation.

%%%%%%%%%%%%%%%%%%%%%% how to use %%%%%%%%%%%%%%%%%%%%%%%%

\section{Usage overview}

\subsection{Calling sequences}

Access to the package requires a {\tt USE} statement:
\begin{verbatim}
   use spral_random_matrix
\end{verbatim}

\noindent
The following procedure is available to the user:
\begin{itemize}
   \item {\tt random\_matrix\_generate()} generates a random matrix to the
      supplied specification.
\end{itemize}

%%%%%%%%%%%%%%%%%%%%%% derived types %%%%%%%%%%%%%%%%%%%%%%%%

\subsection{Derived types}

The user must supply a random number generator state using the type
defined in the \texttt{spral\_random} module. 
The following pseudo-code illustrates how such a generator may be defined.
\begin{verbatim}
      use spral_random, only : random_state
      ...
      type (random_state) :: state
      ...
\end{verbatim}
Further details are given in the documentation for the \texttt{spral\_random}
module. The user may wish to use the routines \texttt{get\_random\_seed()} and
\texttt{set\_random\_seed()} to control the matrix generated.

\subsection{Optional arguments}

We use square brackets {\tt [ ]} to indicate {\it optional} arguments.
In each
call, optional arguments follow the argument {\tt info}.  Since we
reserve the right to add additional optional arguments in future
releases of the code, {\bf we strongly recommend that all optional
arguments be called by keyword, not by position}.

%%%%%%%%%%%%%%%%%%%%%% argument lists %%%%%%%%%%%%%%%%%%%%%%%%

\section{Subroutines}

\subsection{\texttt{random\_matrix\_generate()}}
\textbf{
   To generate an $\texttt{m}\times \texttt{n}$ random matrix with \texttt{nnz}
   non-zero entries,
   \vspace*{0.1cm} \\
   \texttt{ \hspace*{0.2cm}
      call random\_matrix\_generate(state, matrix\_type, m, n, nnz, ptr, row, flag[, stat, val, nonsingular, sort])
   }
}

\noindent
If \texttt{matrix\_type} specifies a
symmetric or skew symmetric matrix, only the lower half matrix will be returned
to the user.

\vspace*{-3mm}
\begin{description}

\item[\texttt{state}] is an \intentinout\ scalar of type {\tt random\_state}.
It is used as the state of the pseudo-random number generator used.

\item[\texttt{matrix\_type}] is an \intentin\ scalar of type default
   {\tt INTEGER}. It specifies the matrix type to be generated, and must be one
   of:
   \begin{itemize}
      \item[\tt 0] Undefined (matrix generated will be unsymmetric/rectangular)
      \item[\tt 1] Rectangular matrix ($\texttt{m}\ne\texttt{n}$)
      \item[\tt 2] Unsymmetric ($\texttt{m}=\texttt{n}$)
      \item[\tt 3] Symmetric positive-definite ($\texttt{m}=\texttt{n}$,
         \texttt{nnz} entries in lower triangle returned, matrix made
         diagonally dominant if \texttt{val} is present, non-singularity
         assured regardless of the value of the optional argument
         {\tt nonsingular})
      \item[\tt 4] Symmetric indefinite ($\texttt{m}=\texttt{n}$,
         \texttt{nnz} entries in lower triangle returned)
      \item[\tt 6] Skew symmetric ($\texttt{m}=\texttt{n}$,
         \texttt{nnz} entries in lower triangle returned)
   \end{itemize}

\item[\texttt{m}] is an \intentin\ scalar of type default {\tt INTEGER} that
   specifies the number of rows in the matrix.
{\bf Restriction:} {\tt m$\geq$1}.

\item[\texttt{n}] is an \intentin\ scalar of type default {\tt INTEGER} that
   specifies the number of columns in the matrix.
{\bf Restriction:} {\tt n$\geq$1}, and consistent with \texttt{matrix\_type}.

\item[\texttt{nnz}] is an \intentin\ scalar of type default {\tt INTEGER} that
   specifies the number of non-zeroes in the matrix.
{\bf Restriction:} {\tt nnz$\geq$1} (and \texttt{nnz}$\geq\min(\texttt{m},\texttt{n})$ if non-singularity requested, or positive-definite).

\item[\texttt{ptr(:)}] is an \intentout\ array of type default {\tt INTEGER}
   and size {\tt n+1}. On exit, {\tt ptr(j)} specifies the position in {\tt row(:)}
   of the first entry in column {\tt j} and {\tt ptr(n+1)$=$nnz+1}.

\item[\texttt{row(:)}] is an \intentout\ array of type default {\tt INTEGER}
   and size {\tt nnz}. On exit, {\tt row(j)} specifies the row to which the
   {\tt j}-th entry belongs.

\item[\texttt{flag}] is an \intentout\ scalar of type default {\tt INTEGER}.
   On exit, it specifies a return code that indicates success with the value
   {\tt 0} or failure with a negative value detailed in Section~\ref{random_matrix:errors}.

\item[\texttt{stat}] is an optional \intentout\ scalar of type default
   {\tt INTEGER}. If present, on exit it returns the value of the Fortran
   \texttt{stat} parameter on the last \texttt{allocate()} call. In particular,
   if \texttt{flag} indicates an allocation failure, it returns further
   information on the failure.

\item[\texttt{val(:)}] is an optional \intentout\ array of type
   {\tt REAL(wp)} and size \texttt{nnz}. On exit, \texttt{val(j)} gives
   the value of the \texttt{j}-th entry. Entries are generated from a uniform
   distribution on the interval $[-1,1]$. In the positive-definite case only,
   diagonal entries are given a value equal to the sum of the off-diagonal
   entries in the row plus a value chosen uniformally at random from the
   interval $(0,1]$.

\item[\texttt{nonsingular}] is an optional \intentin\ scalar of type
   {\tt LOGICAL}. If present with the value {\tt .true.}, the generated matrix
   is guaranteed to have a transversal of size $\min({\tt m}, {\tt n})$. In
   the symmetric or skew symmetric case this will be the natural diagonal. In
   the unsymmetric and rectangular cases a random matching is used. In the
   symmetric positive-definite case, this value is ignored (it is treated as
   {\tt.true.}).
   In all other cases, if {\tt nonsingular} is not present, or is present with
   the value {\tt .false.}, a maximum transversal is not guaranteed and the
   generated matrix may be structurally rank deficient.

\item[\texttt{sort}] is an optional \intentin\ scalar of type {\tt LOGICAL}. If
   present with the value {\tt .true.}, the row entries of the generated matrix
   will be sorted into ascending order within each column.
   Otherwise, if {\tt sort} is not present, or is present with the value
   {\tt .false.}, entries may be returned in a random order within each column.

\end{description}

%%%%%%%%%%%%%%%%%%%%%% Warning and error messages %%%%%%%%%%%%%%%%%%%%%%%%

\section{Return codes} \label{random_matrix:errors}

A successful return is indicated by
{\tt flag} having the value zero.
A negative value is associated with an error message.

Possible negative values are:

\begin{description}
\item{$-$1} An allocation error has occurred. If present, {\tt stat} will
   contain the Fortran {\tt stat} value returned by the failed {\tt allocate()}
   call.
\item{$-$2} An invalid value of {\tt matrix\_type} was supplied.
\item{$-$3} At least one of {\tt m}, {\tt n}, or {\tt nnz} was less than $1$.
\item{$-$4} The (in)equality of {\tt m} and {\tt n} was inconsistent with
   {\tt matrix\_type}.
\item{$-$5} A non-singular matrix was requested, but $\texttt{nnz}<\min(\texttt{m},\texttt{n})$.
\end{description}

\section{Method}

If structural non-singularity is requested, first $\min({\tt m}, {\tt n})$ entries are generated as follows:
\begin{description}
   \item[Unsymmetric or Rectangular] Random permutations of the rows and
      columns are generated. The first $\min({\tt m}, {\tt n})$ entries of
      these permutations are used to specify the entries of a maximum
      transversal.
   \item[Symmetric] The diagonal is added to the matrix explicitly.
\end{description}

The remaining non-zero entries are then assigned to columns uniformally at
random. In the symmetric case, a weighting is used in proportion to the number of
entries below the diagonal. If the selected column for a given non-zero is 
already full, a new random sample is drawn.

Once the number of entries in each column has been determined, and any required
maximum transversal inserted, row indices are determined uniformally at random.
Should a non-zero in that row already be present in the column, a new random
sample is drawn.

In all cases, values are drawn uniformally at random from the range $(-1,1)$. In
the positive-definite case, a post-processing step sums the absolute values of
all the entries in each column and replaces the diagonal with this value.

%%%%%%%%%%%%%%%%%%%%%% EXAMPLE %%%%%%%%%%%%%%%%%%%%%%%%

\section{Example}

The following code generates a random $4 \times 5$ matrix with $8$ non-zeroes
that is non-singular.
\verbatiminput{examples/Fortran/random_matrix.f90}
This produces the following output:
\begin{verbatim}
Generating a   4 x  5 non-singular matrix with   8 non-zeroes
Generated matrix:
Matrix of undefined type, dimension 4x5 with 8 entries.
1:                                         -1.0744E-01   9.1000E-01
2:                             9.5364E-01                1.0912E-01
3:                                          1.1631E-01  -5.8957E-01
4:  -9.0631E-01                                          7.7313E-01
\end{verbatim}

\begin{funders}
   \funder{epsrc}{Funded by EPSRC grant EP/J010553/1}
\end{funders}
