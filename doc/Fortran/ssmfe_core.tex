\packagename{SSMFE\_CORE}
\version{1.0.0}
\versiondate{8 April 2015}

\purpose{
This package
computes extreme (leftmost and/or rightmost)
eigenpairs $\{\lambda_i, x_i\}$ of the following eigenvalue problems:
%
\begin{itemize}
\item 
the standard eigenvalue problem
%
\begin{eqnarray}
\label{ssmfe_core:eqn:evp}
A x = \lambda x,
\end{eqnarray}
%
\item
the generalized eigenvalue problem
%
\begin{eqnarray}
\label{ssmfe_core:eqn:evp.g}
A x = \lambda B x,
\end{eqnarray}
%
\item
the eigenvalue problem
%
\begin{eqnarray}
\label{ssmfe_core:eqn:evp.p}
A B x = \lambda x,
\end{eqnarray}
%
\end{itemize}
%
where 
$A$ and $B$ are {\bf real symmetric} (or {\bf Hermitian}) matrices
and $B$ is {\bf positive definite}.

\texttt{SPRAL\_SSMFE\_CORE}
delegates 
a considerable part of the computation to the user,
which makes its solver procedures rather difficult to use.
For solving problems \EqnRef{ssmfe_core:eqn:evp} and \EqnRef{ssmfe_core:eqn:evp.g},
the user is advised to consider %first 
using
the package \texttt{SPRAL\_SSMFE}, which
offers a simple interface
to \texttt{SPRAL\_SSMFE\_CORE},
or the package \texttt{SPRAL\_SSMFE\_EXPERT},
which delegates less work to the user.
}

\title{Sparse Symmetric Matrix-Free Eigensolver, \\[2pt]
core interface}
%the core procedures package}
\author{
   Evgueni Ovtchinnikov (STFC Rutherford Appleton Laboratory)
}
\pkglang{Fortran}
\spralmaketitle
\thispagestyle{firststyle}

\section*{Major version history}
\begin{description}
\item[2014-11-20 Version 1.0.0] Initial release
\end{description}

%%%%%%%%%%%%%%%%%%%%%% installation %%%%%%%%%%%%%%%%%%%%%%

\section{Installation}
Please see the SPRAL install documentation. In particular note that:
\begin{itemize}
   \item A BLAS library is required.
   \item A LAPACK library is required.
\end{itemize}

\section{Usage overview}

\texttt{SPRAL\_SSMFE\_CORE}
implements a block iterative algorithm
for simultaneous computation of several eigenpairs,
i.e. eigenvalues and corresponding eigenvectors,
of problems \EqnRef{ssmfe_core:eqn:evp}--\EqnRef{ssmfe_core:eqn:evp.p}.
The block nature of this algorithm allows the user
to benefit from highly optimized linear algebra subroutines
and from the ubiquitous multicore architecture
of modern computers.
It also makes this algorithm more reliable
than Krylov-based algorithms employed by e.g. ARPACK
in the presence of clustered eigenvalues.
However, convergence of the iterations may be slow
if the density of the spectrum is high.

Thus, good performance 
(in terms of speed)
is contingent on the following two factors:
(i) the number of desired
eigenpairs must be substantial
(i.e. not less than the number of CPU cores),
and
(ii) the employment of a convergence acceleration technique.
\texttt{SPRAL\_SSMFE\_CORE} allows the user to benefit
from two acceleration techniques:
{\em shift-and-invert} and {\em preconditioning}.
The shift-and-invert technique 
rewrites the eigenvalue problem for a matrix $M$ 
as the eigenvalue problem \EqnRef{ssmfe_core:eqn:evp} for the matrix
$A = (M - \sigma I)^{-1}$,
where $I$ is the identity matrix
and $\sigma$ is a real value near eigenvalues of interest,
and the generalized problem $M x = \mu B x$
as the problem \EqnRef{ssmfe_core:eqn:evp.p}
for $B$ and 
$A = (M - \sigma B)^{-1}$. 
The preconditioning technique applies to 
\EqnRef{ssmfe_core:eqn:evp} and \EqnRef{ssmfe_core:eqn:evp.g} with positive definite $A$ and
requires a matrix or an operator (that is, an algorithm producing a vector $v = T u$ for a given
vector $u$)
$T$, called {\em a preconditioner},
such that the vector
$v = T f$ is an approximation to the solution $u$
of the system $A u = f$.
This technique is more sophisticated
and is likely to be of interest only to experienced users.

For futher detail on the algorithm, see the outline in Section~\ref{ssmfe_core:method}
and associated references.

\texttt{SPRAL\_SSMFE\_CORE} delegates 
more computation to the user than the packages
\texttt{SPRAL\_SSMFE} and \texttt{SPRAL\_SSMFE\_EXPERT}, which are built upon it.
Not only are the storage and operations on all vectors of size $n$
%the problem size, 
user's responsibility, but also
the decision on 
whether a sufficient number of eigenpairs have been computed
to sufficient accuracy.
The amount of computation performed by 
the solver subroutines of \texttt{SPRAL\_SSMFE\_CORE}
and the memory they use are negligible. 
These features facilitate the use of these subroutines
for shared-memory, out-of-core and hybrid computation.

\subsection{Calling sequences}

Access to the package requires a {\tt USE} statement \\ \\
\indent\hspace{8mm}{\tt use \texttt{SPRAL\_SSMFE\_CORE}} \\

\medskip

\noindent The following procedures are available to the user:
%
\vspace{-0.1cm}
\begin{itemize}
\item {\tt ssmfe()} 
computes specified numbers of left- and right-most
eigenvalues and the corresponding eigen\-vec\-tors)
\item {\tt ssmfe\_largest()} 
computes a specified number
of eigenvalues of largest magnitude
and the cor\-res\-pon\-ding eigenvectors
\item {\tt ssmfe\_free()}  should be called after all other calls
are complete. It frees memory referenced by \texttt{keep} and \texttt{inform}.
\end{itemize}

\noindent
The solver procedures \texttt{ssmfe()} and \texttt{ssmfe\_largest()}
must be called repeatedly using
a reverse communication interface.
The procedure \texttt{ssmfe\_free()}
should be called once after the
final call to 
a solver procedure
to free all memory that has been internally
allocated by
the solver procedure.

\subsection{Package types} 

\texttt{INTEGER} denotes default \texttt{INTEGER}, and \texttt{REAL} denotes
\texttt{REAL(kind=kind(0d0))}. We use the term {\bf package type} to mean
\texttt{REAL(kind=kind(0d0))} for calls to the double precision real valued
interface, and to mean \texttt{COMPLEX(kind=kind(0d0))} for calls to the double
precision complex valued interface.

\subsection{Derived types}

For each problem, the user must employ the derived types defined by the
module to declare scalars of the types 
{\tt ssmfe\_rcid} (real version) or 
{\tt ssmfe\_rciz} (complex version), 
{\tt ssmfe\_core\_keep},
{\tt ssmfe\_core\_options} and 
{\tt ssmfe\_inform}.
The following pseudocode illustrates this.
\begin{verbatim}
      use SPRAL_SSMFE_CORE
      ...
      type (ssmfe_rcid        ) :: rcid
      type (ssmfe_core_keep   ) :: keep
      type (ssmfe_core_options) :: options
      type (ssmfe_inform      ) :: inform
      ...
\end{verbatim}
The components of {\tt ssmfe\_options} and {\tt ssmfe\_inform} are explained
in Sections~\ref{ssmfe_core:type:options} and \ref{ssmfe_core:type:inform}. Components
of \texttt{ssmfe\_rcid} are used for the reverse communication interface and
are explained in Section~\ref{ssmfe_core:fn:ssmfe}.
The components of \texttt{keep} are used to pass data between repeated calls to
the subroutines and must not be altered by the user.

\section{Argument lists}

\subsection{\texttt{ssmfe()} and \texttt{ssmfe\_largest()}} \label{ssmfe_core:fn:ssmfe}

{\bf
To compute specified numbers
of leftmost and rightmost eigenvalues 
and corresponding eigenvectors,
the following call must be made repeatedly:
}

\medskip
{\tt call
ssmfe( rci, problem, left, right, m, lambda, rr, ind, keep, options, inform )
}

\medskip
\noindent
{\bf
To compute a specified number %amount 
of eigenvalues of largest magnitude 
and corresponding eigenvectors,
the following call must be made repeatedly:
}

\medskip
{\tt call
ssmfe\_largest( rci, problem, nep, m, lambda, rr, ind, keep, options, inform )
}

\medskip
\noindent
To use the solver procedures,
the user needs to maintain a workspace {\tt W} containing
{\tt kw + 1} blocks of {\tt m} vectors of size {\tt n}.
A value {\tt kw = 7} is always sufficient. 
However, if {\tt options\%minAprod} $=$ {\tt .false.}
and either {\tt options\%minBprod} $=$ {\tt .false.} or 
the standard eigenvalue problem \EqnRef{ssmfe_core:eqn:evp} is solved,
then {\tt kw = 3} is sufficient; 
if 
{\tt options\%minAprod} $=$ {\tt .true.} and
either {\tt options\%minBprod} $=$ {\tt .true.} or
\EqnRef{ssmfe_core:eqn:evp.p} is solved,
then {\tt kw} must be at least {\tt 7};
otherwise {\tt kw = 5} is sufficient.
Solver procedures
use indices {\tt 1} to {\tt m} 
to refer to vectors inside each block
and indices {\tt 0} to {\tt kw} 
to refer to particular blocks.
The first (zero-indexed) block holds the eigenvector approximations:
the user must fill this block with 
{\tt m} linearly independent vectors before the first call
to a solver procedure.

The number of desired eigenpairs may exceed {\tt m}:
whenever converged eigenpairs have been detected,
a solver procedure reports the indices of these eigenpairs
and they must be moved by the user
to a separate eigenvectors' storage {\tt X}.

When $B \ne I$,
it is expedient to 
have %another 
storage {\tt BX}
for the $B$-images of the converged eigenvectors,
i.e. {\tt BX = B*X}.

To simplify the description of the %advanced RCI's,
reverse communication interface,
below we assume that an array
{\tt W(n,~m,~0:kw)} of package type
is used as a workspace,
and that arrays {\tt X(n, mep)} and {\tt BX(n, mep)} of package type
are used for storing the computed eigenvectors
and their $B$-images,
where {\tt mep} is not less than the number of desired eigenpairs.
The transpose (real or complex, depending on the package type)
of a matrix {\tt H} 
is denoted by {\tt H}$^\prime$.

\medskip
\noindent
The meaning of the arguments of the solver procedures is as follows.

\begin{description}
%
\item[\texttt{rci}] is is an \intentinout\ scalar  of type
{\tt ssmfe\_rcid} in real version and
{\tt ssmfe\_rciz} in complex version.
Before the first call, {\tt rci\%job} must be set to {\tt 0}.
No other values may be assigned to {\tt rci\%job} by the user.
After each call,
the value of {\tt rci\%job} must be inspected by the user's code
and the appropriate action taken: 
%(see below for details).
%The following values of {\tt rci\%job}
%are common to all solver procedures 
%and require the same action: 
%
\begin{description}
%
\item[\texttt{-3}]: fatal error return, the computation must be terminated;
%
\item[\texttt{-2}]: 
not all desired eigenpairs converged to required accuracy,
see Section~\ref{ssmfe_core:errors};
%
\item[\texttt{-1}]: the computation is complete and successful.
%
\item[\texttt{~1}]: the user must compute $V = A U$, where

\hspace{8mm}
$U=$ {\tt W(:, ix:jx, rci\%kx)}, 
~with~ {\tt ix} $=$ {\tt rci\%jx} 
~and~
{\tt jx} $=$ {\tt ix + rci\%nx - 1},

\hspace{8mm}
$V=$ {\tt W(:, iy:jy, rci\%ky)},
~with~ {\tt iy} $=$ {\tt rci\%jy} 
~and~
{\tt jy} $=$ {\tt iy + rci\%nx - 1}.
%
\item[\texttt{~2}]: the user must
compute $V = T U$ if preconditioning is used
or copy $U$ to $V$ otherwise,
where $U$ and $V$ are as for {\tt rci\%job = 1}.
%
\item[\texttt{~3}]:
the user must compute $V = B U$,
where $U$ and $V$ are as for {\tt rci\%job = 1}.
%
\item[\texttt{~4}]:
for each {\tt i = 1,\ldots,m}
such that {\tt inform\%converged(i)} is zero,
the user must set {\tt inform\%converged(i)}
to a positive value if the residual norm
{\tt inform\%res\_norms(i)} and
the estimated eigenvalue and eigenvector errors
{\tt inform\%err\_lambda(i)} and
{\tt inform\%err\_x(i)} are small enough
for this eigenpair to be deemed as converged.
%
\item[\texttt{~5}]: the user must save the converged eigenvectors
to the eigenvector storage {\tt X}
and, optionally, 
for problems \EqnRef{ssmfe_core:eqn:evp.g} and \EqnRef{ssmfe_core:eqn:evp.p},
save their $B$-images.
The converged eigenvectors are columns of the $n\times m$ matrix
{\tt W(:,:,rci\%kx)} and their $B$-images are respective columns of
{\tt W(:,:,rci\%ky)}
that are identified by
{\tt rci\%i, rci\%jx} and {\tt rci\%nx}
as follows:
if {\tt rci\%i > 0}, then the column numbers
run from {\tt rci\%jx} to {\tt rci\%jx + rci\%nx - 1},
and if {\tt rci\%i < 0}, then they run
from {\tt rci\%jx - rci\%nx + 1} to {\tt rci\%jx}.
%
\item[\texttt{11}]:
if {\tt rci\%i = 0}, then
the user  must perform a copy $V \leftarrow U$, 
where $U$ and $V$ are as for {\tt rci\%job = 1},
otherwise the columns of {\tt W(:,:,rci\%kx)}
and {\tt W(:,:,rci\%ky)}
(if {\tt rci\%kx} $\not=$ {\tt rci\%ky}) 
must be reordered using
the index array {\tt ind(:)} so that
the column {\tt ind(j)} becomes column {\tt j}
for {\tt j = 1,\ldots,rci\%nx}.
%
\item[\texttt{12}]:
for each
{\tt i = 0, 1,..., rci\%nx - 1}, 
the user must compute the dot product of
the columns 

\hspace{8mm}
{\tt W(:, rci\%jx + i, rci\%kx)} 

and

\hspace{8mm}
{\tt W(:, rci\%jy + i, rci\%ky)}

and place it in 

\hspace{8mm}
{\tt rr(rci\%i + i, rci\%j + i, rci\%k)}.
%
\item[\texttt{13}]: 
if {\tt rci\%kx} $=$ {\tt rci\%ky}, then
for each
{\tt i = 0, 1,..., rci\%nx - 1}, 
the user must perform the scaling

\hspace{8mm}
{\tt W(:, rci\%jx + i, rci\%kx) = W(:, rci\%jx + i, rci\%kx)$/s_i$},

where $s_i$ is the 2-norm of the column 
{\tt W(:, rci\%jx + i, rci\%kx)},
otherwise the user must perform the scalings

\hspace{8mm}
{\tt W(:, rci\%jx + i, rci\%kx) = W(:, rci\%jx + i, rci\%kx)$/s_i$}

\hspace{8mm}
{\tt W(:, rci\%jy + i, rci\%ky) = W(:, rci\%jy + i, rci\%ky)$/s_i$},

where $s_i$ is the square root of the dot product of 
the columns 
{\tt W(:, rci\%jx + i, rci\%kx)} and
{\tt W(:, rci\%jy + i, rci\%ky)}.
No scaling is to be applied to zero columns.
%
\item
{\tt 14}: 
for each {\tt i = 0, 1,..., rci\%nx - 1}, 
the user must perform axpy-updates:

\hspace{8mm}
{\tt 
W(:, rci\%jy + i, rci\%ky) = 
W(:, rci\%jy + i, rci\%ky) + 
}

\hspace{12mm}
{\tt
rr(rci\%i + i, rci\%j + i, rci\%k) * W(:, rci\%jx + i, rci\%kx)}.
%
\item
{\tt 15}: the user must perform the matrix multiplication:

\hspace{8mm}
{\tt rr(rci\%i $:$ rci\%i + rci\%nx - 1, rci\%j $:$ rci\%j + rci\%ny-1, rci\%k) =}

\hspace{12mm}
{\tt rci\%alpha *
W(:, rci\%jx $:$ rci\%jx + rci\%nx - 1, rci\%kx)' *}

\hspace{16mm}
{\tt W(:, rci\%jy $:$ rci\%jy + rci\%ny - 1, rci\%ky) +}

\hspace{12mm}
{\tt 
rci\%beta *
rr(rci\%i $:$ rci\%i + rci\%nx - 1, rci\%j $:$ rci\%j + rci\%ny - 1, rci\%k)}.

%where {\tt W'} denotes the transpose of {\tt W};
%
\item
{\tt 16}: the user must perform the matrix multiplication:

\hspace{8mm}
{\tt W(:, rci\%jy $:$ rci\%jy + rci\%ny - 1, rci\%ky) =}

\hspace{12mm}
{\tt rci\%alpha *
W(:, rci\%jx $:$ rci\%jx + rci\%nx - 1, rci\%kx) *}

\hspace{16mm}
{\tt rr(rci\%i $:$ rci\%i + rci\%nx - 1, 
rci\%j $:$ rci\%j + rci\%ny - 1, rci\%k) + }

\hspace{12mm}
{\tt rci\%beta * W(:, rci\%jy $:$ rci\%jy + rci\%ny - 1, rci\%ky)}.
%
\item
{\tt 17}: the user must perform the multiplication:


\hspace{8mm}
{\tt W(:, rci\%jx $:$ rci\%jx + rci\%ny - 1, rci\%kx) =}

\hspace{12mm}
{\tt W(:, rci\%jx $:$ rci\%jx + rci\%nx - 1, rci\%kx) *}

\hspace{16mm}
{\tt rr(rci\%i $:$ rci\%i + rci\%nx - 1, 
rci\%j $:$ rci\%j + rci\%ny - 1, rci\%k)}.

{\tt W(:, rci\%jy $:$ rci\%jy + rci\%ny - 1, rci\%ky)}
can be used as a workspace.
%
\item
{\tt 21}: 
the user must $B$-orthogonalize
the columns of {\tt W} specified by
{\tt rci\%nx}, {\tt rci\%jx} and {\tt rci\%kx} 
to all vectors stored in {\tt X}
by solving the system

\hspace{8mm}
{\tt (X' * BX) Q = X' * W(:, rci\%jy $:$ rci\%jy + rci\%nx - 1, rci\%ky)}

for {\tt Q} and updating

\hspace{8mm}
{\tt W(:, rci\%jx $:$ rci\%jx + rci\%nx - 1, rci\%kx) =}

\hspace{12mm}
{\tt W(:, rci\%jx $:$ rci\%jx + rci\%nx - 1, rci\%kx) - X * Q}.

For problems \EqnRef{ssmfe_core:eqn:evp.g} and \EqnRef{ssmfe_core:eqn:evp.p},
the respective columns of {\tt W(:,:,rci\%ky)},
which store $B$-images of the respective columns of {\tt W(:,:,rci\%kx)},
must be updated
accordingly, either by applying {\tt B} to these vectors
or using the columns of {\tt BX}, i.e.

\hspace{8mm}
{\tt W(:, rci\%jy $:$ rci\%jy + rci\%nx - 1, rci\%ky) =}

\hspace{12mm}
{\tt W(:, rci\%jy $:$ rci\%jy + rci\%nx - 1, rci\%ky) - BX * Q};
%
\item
{\tt 22}: 
the user must solve the system

\hspace{8mm}
{\tt (X' * BX) Q = 
X' * W(:, rci\%jx $:$ rci\%jx + rci\%nx - 1, rci\%kx)}

for {\tt Q}  and perform the update

\hspace{8mm}
{\tt W(:, rci\%jx $:$ rci\%jx + rci\%nx - 1, rci\%kx) =}

\hspace{12mm}
{\tt W(:, rci\%jx $:$ rci\%jx + rci\%nx - 1, rci\%kx) - BX * Q},

where {\tt X} and {\tt BX} are same as in 
the case {\tt rci\%job = 21}
(in the case of problem \EqnRef{ssmfe_core:eqn:evp},
{\tt rci\%job = 21} and {\tt 22} 
require exactly the same computation).
%
\item
{\tt 999}: 
If {\tt rci\%k > 0}, then
a restart, normally with a larger block size {\tt m},
is suggested with the aim of achieving better convergence.
If the suggestion is accepted, the user must compute
the new block size as {\tt m = rci\%nx + k + l},
where {\tt k $\ge$ rci\%i} and {\tt l $\ge$ rci\%j},
reallocate the workspace array {\tt W}
if the new block size is different from the old one,
and set {\tt rci\%i = 0} and {\tt rci\%j = 0}.
The first block {\tt W(:,:,0)} of the new
workspace should retain the vectors 
{\tt W(:,i:j,0)},
where {\tt i = rci\%jx} and {\tt j = i + rci\%nx - 1},
from the old workspace.
The remaining {\tt m - rci\%nx} columns of {\tt W(:,:,0)}
must be filled
with arbitrary vectors that are linearly independent from 
the converged eigenvectors and such that
the entire set of the columns of {\tt W(:,:,0)}
is linearly independent.
If the restart is not acceptable
(e.g. the new block size exceeds a certain limit set by the user), 
then nothing needs to be done.
%
\end{description}
%
{\bf Restriction:} 
{\tt rci\%job = 0}, 
{\tt rci\%i = 0} and
{\tt rci\%j = 0} 
are the only %values that can be assigned 
assignments to the components of {\tt rci}
that can be done
by the user.
The first one can only be done before the first call.
The other two can only be done if
{\tt rci\%job = 999} and {\tt rci\%k > 0}.
%
\item[\texttt{problem}] is an \intentin\ scalar of type {\tt INTEGER} that
specifies the problem to be solved:
if {\tt problem} is zero, then \EqnRef{ssmfe_core:eqn:evp} is solved,
if it is positive then \EqnRef{ssmfe_core:eqn:evp.g} is solved,
otherwise \EqnRef{ssmfe_core:eqn:evp.p} is solved.
%
\item[\texttt{left}] is an \intentin\ scalar of type default \texttt{INTEGER}
that specifies the number of desired leftmost eigenvalues.
On return with {\tt rci\%job = 5}, can be set to zero if a
sufficient number of leftmost eigenvalues have been computed.
%
\item[\texttt{right}] is an \intentin\ scalar of type default \texttt{INTEGER}
that specifies the number of desired rightmost eigenvalues.
On return with {\tt rci\%job = 5}, can be set to zero if
sufficient number of rightmost eigenvalues have been computed.
%
\item[\texttt{nep}] is an \intentin\ scalar of type default \texttt{INTEGER}
that specifies the number of desired largest eigenvalues.
%
\item[\texttt{lambda(:)}] is an \intentinout\
array of type \texttt{REAL}
and size {\tt m} 
that is
used to return the computed eigenvalues.
After a successful completion of the computation
it contains eigenvalues in ascending order.
This array must not be changed by the user.
%
\item[\texttt{m}] is an \intentin\ scalar of type {\tt INTEGER} that
holds the block size of the user's workspace {\tt W}. 
{\bf Restriction:} %\\{\tt m $<$ n},
if both {\tt left} and {\tt right} are non-zero
or \texttt{ssmfe\_largest()} is called then {\tt m $>1$},
otherwise {\tt m $>0$}.
%
\item[\texttt{rr(:,:,:)}] is an \intentinout\
work array of package type,
and dimensions {\tt 2*m}, {\tt 2*m} and {\tt 3}.
It must only be changed by the user when
instructed to do so by 
{\tt rci\%job}.
%
\item[\texttt{ind(:)}] is an \intentinout\ 
array of default integer type, and size at least {\tt m}. 
It must not be changed by the user.
It is used for reordering the columns of some blocks of {\tt W}.
%
\item[\texttt{keep}] is an \intentinout\ scalar of type 
{\tt ssmfe\_core\_keep} that holds private data. 
%
\item[\texttt{options}] is an \intentin\  scalar  of type {\tt ssmfe\_core\_options}.
Its components offer the user a range of options,
see Section~\ref{ssmfe_core:type:options}.
It must not be changed by the user between calls.
%
\item[\texttt{inform}] is an \intentinout\ scalar of type 
{\tt ssmfe\_inform}. Its components provide information about the execution
of the subroutine, see Section~\ref{ssmfe_core:type:inform}.
It must not be changed by the user.
%
\end{description}

\subsection{ssmfe\_free()}

{\bf
At the end of the computation, the memory 
allocated by the solver procedures
should be released
by making the following subroutine call:
}

\medskip

\hspace{8mm} {\tt ssmfe\_free( keep, inform )}

\begin{description}
%
\item[\texttt{keep}] is an \intentinout\ scalar of type 
{\tt ssmfe\_core\_keepk}, optional. 
On exit, its components that are allocatable arrays will have been deallocated. 
%
\item[\texttt{inform}] is an \intentinout\ scalar of type {\tt ssmfe\_inform}, 
optional.
On exit, its components that are allocatable arrays will have been deallocated. 
%
\end{description}

\section{Derived types}
\subsection{\texttt{ssmfe\_core\_options}} \label{ssmfe_core:type:options}

The derived data type {\tt ssmfe\_core\_options} is used to specify
the options used within {\tt SSMFE\_CORE}. The components, that are
automatically given default values in the definition of the type, are:

\begin{description}
%
\item[\texttt{err\_est}] 
is a scalar of type default \texttt{INTEGER} that
defines which error estimation scheme 
for eigenvalues and eigenvectors
is to be used by the stopping criterion.
Two schemes are implemented.
If {\tt err\_est = 1}, residual error bounds are used,
namely,
a modified Davis-Kahan estimate for the eigenvector error
and
the Lehmann bounds for the eigenvalue error.
(see Section~\ref{ssmfe_core:sec:err_est}).
If {\tt err\_est = 2}, 
then the eigenvector and eigenvalue errors
are estimated by analyzing the convergence curve
for the eigenvalues (see Section~\ref{ssmfe_core:sec:err_est}).
The default is {\tt err\_est = 2}.
{\bf Restriction:} {\tt err\_est = 1 {\rm or} 2}.
%
\item[\texttt{extra\_left}]
is a scalar of type default \texttt{INTEGER} that
holds the number of extra approximate eigenvectors
corresponding to leftmost eigenvalues
that are of no interest to the user
and are iterated solely to enhance convergence.
The default is {\tt extra\_left = 0}.
{\bf Restriction:} {\tt extra\_left $\ge$ 0}.
%
\item[\texttt{extra\_right}]
is a scalar of type default \texttt{INTEGER} that
holds the number of extra approximate eigenvectors
corresponding to rightmost eigenvalues
that are of no interest to the user
and are iterated solely to enhance convergence.
The default is {\tt extra\_right = 0}.
{\bf Restriction:} {\tt extra\_right $\ge$ 0}.
%
\item[\texttt{minAprod}] is a scalar of type default \texttt{LOGICAL} that
determines whether the number of multiplications by $A$ 
is to be reduced at the expense of memory. 
If ${\tt minAprod = .false.}$, 
on each iteration three returns to the user
with {\tt rci\%job = 1} are
made for multiplications of {\tt rci\%nx} vectors by $A$.
Otherwise,  only one such return is made at each iteration but 
the number {\tt kw} of blocks in the user's work array {\tt W} 
must be increased by {\tt 2}.
The default is {\tt minAprod = .true.}.
{\bf Restriction:} {\tt minAprod = .true.} if {\tt problem < 0}.
%
\item[\texttt{minBprod}] is a scalar of type default \texttt{LOGICAL} that
determines whether the number of multiplications by $B$ 
is to be reduced at the expense of memory. 
If ${\tt minBprod = .false.}$, 
on each iteration at least three returns to the user
with {\tt rci\%job = 3} are
made for multiplications of {\tt rci\%nx} vectors by $B$.
Otherwise,  only one such return is made at each iteration but 
the number {\tt kw} of blocks in the user's work array {\tt W} 
must be increased by {\tt 2}.
The default is {\tt minBprod = .true.}.
%
\item[\texttt{min\_gap}] 
is a scalar of type {\tt REAL}\ that
controls the restart.
If the relative distance between the last eigenvalue of interest
on either margin of the spectrum and the rest of the spectrum
is smaller than {\tt min\_gap},
the solver procedure suggests restart
(cf. {\tt rci\%job $= 999$}).
The values 
{\tt rci\%i}
and
{\tt rci\%j}
are set to the numbers of eigenvalues on the left and right
margin of the spectrum that are too close to the eigenvalues
of interest, causing slow convergence.
The default is {\tt min\_gap = 0}, i.e.
by default no restart is ever suggested.
{\bf Restriction:} {\tt $0 \le$ min\_gap $\le 1$}.
%
\item[\texttt{cf\_max}]
is a scalar of type {\tt REAL} that
is used to detect the convergence stagnation
based on 
the estimated asymptotic convergence factor 
$q_{ij}$ given by \EqnRef{ssmfe_core:eqn:acf}
(see \S\ref{ssmfe_core:sec:err_est}).
If $q_{ij}$ is greater than {\tt cf\_max} for $i > 5$,
the eigenpair is marked as stagnated by setting
{\tt inform\%converged(j)} to a negative value.
The default is {\tt cf\_max = 1}, i.e.
the estimated asymptotic convergence factor
is not used for stagnation detection.
{\bf Restriction:} {\tt $0.5 \le$ cf\_max $\le 1$}.
%
\end{description}

\subsection{\texttt{ssmfe\_infrom}} \label{ssmfe_core:type:inform}

The derived data type {\tt ssmfe\_inform} is used
to hold information from the execution of
the solver procedures.
The components are:

\begin{description}
%
\item[\texttt{converged}] is a rank-1 allocatable array of type 
default \texttt{INTEGER} that is allocated to have size {\tt m} 
on a call with {\tt rci\%job = 0} or {\tt 999}.
If, on some iteration {\tt i}, an eigenpair ({\tt lambda(j), X(j)})
has been accepted as converged, then {\tt converged(j) = i}; 
if the convergence stagnated, then {\tt converged(j) = -i}; 
otherwise {\tt converged(j) = 0}.
%
\item[\texttt{err\_lambda}] is a rank-1 allocatable array of type 
{\tt REAL} that is allocated to have size {\tt m} 
on a call with {\tt rci\%job = 0} or {\tt 999}.
{\tt err\_lambda(i)} contains 
the estimated eigenvalue error
for the approximate eigenvalue {\tt lambda(i)}
if {\tt info\%converged(i)} is non-zero,
and {\tt -1.0} otherwise.
%
\item[\texttt{err\_x}] is a rank-1 allocatable array of type {\tt REAL}.
This array is allocated to have size {\tt m} 
on a call with {\tt rci\%job = 0} or {\tt 999},
and is used for storing the eigenvector errors
in the same way as {\tt err\_lambda(:)} is used
for storing the eigenvalue errors.
%
\item[\texttt{flag}] is a scalar of type default \texttt{INTEGER}
that is used as an error flag.
If a call is successful, {\tt flag} has value {\tt 0}.
A nonzero value of {\tt flag} indicates an error or a warning
(see Section~\ref{ssmfe_core:errors}).
%
\item[\texttt{iteration}] is a scalar of type default \texttt{INTEGER}
that holds the number of iterations 
since the previous {\tt rci\%job = 0} or {\tt rci\%job = 999} call.
%
\item[\texttt{residual\_norms}] is a rank-1 allocatable array of type
{\tt REAL} that is allocated to have size {\tt m} 
on a call with {\tt rci\%job = 0} or {\tt 999}.
On returns with 
{\tt rci\%job = 4} or {\tt 5}, {\tt res\_norms(i)} 
contains the Euclidean norm of the residual
for {\tt lambda(i), X(i)}.
%
\item[\texttt{stat}]
is a scalar of type default \texttt{INTEGER}
that holds the allocation status
(see Section~\ref{ssmfe_core:errors}).
%
\end{description}

\section{Error flags}

\label{ssmfe_core:errors}

A successful return from 
ssmfe\ and
ssmfe\_largest\
is indicated 
by {\tt inform\%flag$=$0}.
A negative value indicates an error, a positive value indicates a warning.

Possible negative values of {\tt inform\%flag}
are as follows:
%
\begin{description}
%
\item{~~-1}
\hskip 9pt
Block size {\tt m} is out-of-range.
%
\item{~~-2}
\hskip 9pt
Incorrect value of {\tt rci\%job}. 
%
\item{~~-3}
\hskip 9pt
Incorrect value of {\tt options\%err\_est}. 
%
\item{~~-4}
\hskip 9pt
Incorrect value of {\tt options\%minAprod}.
%
\item{~~-5}
\hskip 9pt
Incorrect value of 
{\tt options\%extra\_left} or
{\tt options\%extra\_right}.
%
\item{~~-6}
\hskip 9pt
Incorrect value of {\tt options\%min\_gap}. 
%
\item{~~-7}
\hskip 9pt
Incorrect value of {\tt options\%cf\_max}. 
%
\item{~-11}
\hskip 7pt
Incorrect value of {\tt left} (for {\tt ssmfe})
or {\tt nep} (for {\tt ssmfe\_largest}).
%
\item{~-12}
\hskip 7pt
Incorrect value of {\tt right}. 
%
\item{-100}
\hskip 4pt
Not enough memory;
{\tt inform\%stat} contains the value of the Fortran {\tt stat} parameter.
%
\item{-200}
\hskip 4pt
$B$ is not positive definite or initial eigenvectors are linearly dependent.
%or an unauthorized change has been made to
%{\tt rr} or {\tt W}.
%
\end{description}

%Possible positive values  are: 
%
%\begin{description}
%\item{1}
%\hskip 9pt
%A positive value 
The value {\tt inform\%flag = 1}
indicates that
the iterations have been terminated because no further improvement
in accuracy is possible (this may happen if $B$ or the preconditioner is
not positive definite, or if the components of the residual vectors
are so small that the round-off
errors make them essentially random).
%
%\end{description}

\section{Method}
\label{ssmfe_core:method}

\subsection{The algorithm}

The algorithm implemented by {\tt ssmfe()} and {\tt ssmfe\_largest()}
is based on the Jacobi-conjugate preconditioned gradients (JCPG) method.
The outline of the algorithm,
assuming for simplicity
that {\tt 0 < left $\le$ m} and {\tt right = 0}
and using Matlab notation, is as follows.

\begin{itemize}
\item 
{\bf Initialization.}
Perform the Rayleigh-Ritz procedure
in the trial subspace spanned by the columns of 
an ${\tt n} \times {\tt m}$ matrix {\tt X} i.e. compute 
\begin{itemize}
\item 
{\tt L = X'*A*X},
if {\tt problem $\ge$ 0}, 
and
{\tt L = X'*B*A*B*X} otherwise,
\item
{\tt M = X'*B*X}
({\tt B=I} if {\tt problem = 0}),
\end{itemize}
% 
and solve the generalized eigenvalue problem
for the matrix pencil ${\tt L - t*M}$, i.e.
compute an ${\tt m} \times {\tt m}$ matrix {\tt Q}
such that {\tt Q'*M*Q} is the identity matrix
and {\tt D = Q'*L*Q} is a diagonal matrix.
Compute {\tt X = X*Q} and set {\tt Z = F = []}.
%
\item
{\bf Main loop.} 

{\tt DO}
\begin{enumerate}
%
\item
\label{ssmfe_core:method:step:do.R}
If {\tt problem$\ge$0}, compute the residual matrix
{\tt R = A*X - B*X*D},
where {\tt D} is a  diagonal matrix with entries
%
\begin{equation}
\label{ssmfe_core:eqn:rq}
\mbox{\tt D(j,j)}=
\frac{\mbox{\tt X(:,j)'*A*X(:,j)}}
{\mbox{\tt X(:,j)'*B*X(:,j)}}
\end{equation}
%
else compute {\tt R = A*B*X - X*D},
where {\tt D} is a diagonal matrix with entries
%
\begin{equation}
\label{ssmfe_core:eqn:rq.AB}
\mbox{\tt D(j,j)}=
\frac{\mbox{\tt X(:,j)'*B*A*B*X(:,j)}}
{\mbox{\tt X(:,j)'*B*X(:,j)}}.
\end{equation}
%
\item
\label{ssmfe_core:method:step:do.ortho.R}
Perform the orthogonalization of {\tt R} 
to constraints {\tt C} by updating
{\tt R = R - B*C*(C'*R)}.
%
\item
\label{ssmfe_core:method:step:do.save}
If {\tt options\%err\_est $= $1},  compute {\tt R'*B*R}
and use it to compute error bounds;
otherwise only compute the diagonal of this matrix
and use to compute error bounds.
Test for converged eigenpairs
and move converged eigenvectors from {\tt X} to {\tt C}
and reduce {\tt m} accordingly.
Exit the main loop if {\tt X = []}.
%
\item
If {\tt problem} is non-negative,
compute the preconditioned gradient
matrix {\tt Y = T*R}.
%
\item
\label{ssmfe_core:method:step:do.JC}
If {\tt Z} is not empty, conjugate {\tt Y} to {\tt Z}, i.e.
%
\begin{enumerate}
\item 
if {\tt problem} is non-negative,
then compute {\tt P = Z'*A*Y}, 
otherwise compute {\tt P = Z'*B*A*B*Y};
\item
compute {\tt S = Z'*B*Y};
\item
update {\tt Y = Y + Z*H},
where
\begin{equation}
\label{ssmfe_core:eqn:JC}
\mbox{\tt H(i,j)}=
\frac{\mbox{\tt P(i,j) - S(i,j)*D(j,j)}}
{\mbox{\tt F(i,i) - D(j,j)}}
\end{equation}
\end{enumerate}
where {\tt F} is described in step~\ref{ssmfe_core:method:step:do.update}.
%
\item
\label{ssmfe_core:method:step:do.ortho.Y}
Perform orthogonalization of the search direction matrix {\tt Y} 
to constraints {\tt C} by updating\\
{\tt Y = Y - C*(C'*B*Y)}.
%
\item
\label{ssmfe_core:method:step:do.cleanup.Y}
{\tt B}-normalize the columns of {\tt Y} 
and reorder them
so that the {\tt B}-norm of the projection of
{\tt Y(:,j)} onto the linear span of
{\tt X} and {\tt Y(:,1:j-1)} is
an increasing function of {\tt j}.
Compute {\tt M = [X Y]'*B*[X Y]}.
If the condition number of {\tt M}
is greater than the allowed limit
of $10^4$
then start removing the last columns
of {\tt Y} and {\tt M} and respective
rows of {\tt M} until
the condition number falls below the limit.
%
\item
\label{ssmfe_core:method:step:do.update}
Perform the Rayleigh-Ritz procedure
in the linear span of columns of {\tt [X Y]}.
Update {\tt X} by selecting
Ritz vectors corresponding to the leftmost 
{\tt m} Ritz values,
and place the remaining Ritz vectors into {\tt Z}
and corresponding Ritz values onto the diagonal of {\tt F}.
%
\end{enumerate}

{\tt END DO}

\end{itemize}

The orthogonalization to constraints on step~\ref{ssmfe_core:method:step:do.ortho.R}
ensures that the algorithm deals with 
the residuals for the constrained problem
rather than with those for the unconstrained one,
which may not go to zeros
if the constraints are not close enough to exact eigenvectors.

\EqnRef{ssmfe_core:eqn:JC} ensures  optimality of
the new search direction vector {\tt Y(:,j)},
notably, the search for the minimum
of the Rayleigh quotient \EqnRef{ssmfe_core:eqn:rq} or,
in the case of problem \EqnRef{ssmfe_core:eqn:evp.p}, \EqnRef{ssmfe_core:eqn:rq.AB}
in the direction of this vector
produces asymptotically smallest possible value.

The search directions cleanup procedure
employed on step~\ref{ssmfe_core:method:step:do.cleanup.Y}
ensures that the convergence of iterations 
is not damaged by
the computational errors
of the LAPACK eigensolver {\tt \_SYGV},
in the Rayleigh-Ritz procedure of step~\ref{ssmfe_core:method:step:do.update}.
The accuracy of {\tt \_SYGV} is affected
by the condition number of {\tt M}.
If the latter is very large,
poor accuracy in the computed Ritz vectors may
lead to convergence failure.
The ordering of {\tt Y(:,j)} ensures
that the `least important' search direction
is dropped if the condition number of {\tt M}
is unacceptably large.
In practice, the loss of search direction is 
a rare 
and does not lead to convergence problems.

If the number of sought eigenpairs exceeds {\tt m}, then
{\tt m} is not reduced on step~\ref{ssmfe_core:method:step:do.save}.
Instead, the approximate eigenvectors
moved to {\tt C} are replaced with vectors from {\tt Z}.

\subsection{Error estimation}

\label{ssmfe_core:sec:err_est}

\subsubsection{Standard problem}

If {\tt options\%err\_est} {\tt =} {\tt 1}, 
the error estimates for the eigenvalues are based on 
the eigenvalues of a matrix of the form
%
\begin{eqnarray}
\hat A = \tilde\Lambda_k - S_k^T S_k,
\end{eqnarray}
%
where $\tilde\Lambda_k$ is a diagonal matrix with
the $k-1$ leftmost Ritz values $\tilde\lambda_j$
on the diagonal,
and the columns of $S_k$ are the respective
residual vectors $r_j = A \tilde x_j - \tilde\lambda_j \tilde x_j$
divided by $\sqrt{\lambda_k - \tilde\lambda_j}$.
If $k$ is such that
$\tilde\lambda_{k-1} < \lambda_k$,
then the eigenvalues of $\hat A$ are
the left-hand side bounds for
eigenvalues $\lambda_i$,
and thus
the difference $\tilde\lambda_j - \hat\lambda_j$ estimates
the eigenvalue error $\tilde\lambda_j - \lambda_j$.
The unknown  $\lambda_k$ is replaced by $\tilde\lambda_k$,
and select the maximal $k \le m$ for which
the distance between $\tilde\lambda_{k-1}$ and $\tilde\lambda_k$
exceeds the sum of the absolute error tolerance for eigenvalues
and the Frobenius norm of the matrix formed by the residuals
$r_j, j = 1,\ldots,k-1$.
If  $\tilde\lambda_j - \hat\lambda_j$
is close to the machine accuracy, it may be too polluted
by round-off errors to rely upon.
In such case, we use instead
%
\begin{eqnarray}
\label{ssmfe_core:eqn:aL}
\tilde\lambda_j - \lambda_j \le \delta_j \approx
\frac{\|r_j\|^2}{\tilde\lambda_k - \lambda_j}.
\end{eqnarray}

The eigenvector errors are estimated based on 
the Davis-Kahan inequality:
%
\begin{eqnarray}
\label{ssmfe_core:eqn:DK}
\min_{x \in \mathcal{X}_{k-1}}
\sin\{\tilde x_j; x\} \le
\frac{\|r_j\|}{\lambda_k - \tilde\lambda_j} \approx
\frac{\|r_j\|}{\tilde\lambda_k - \tilde\lambda_j},
\end{eqnarray}
%
where $\mathcal{X}_{k-1}$ is the invariant subspace 
corresponding to $k-1$ leftmost eigenvalues.

If {\tt options\%err\_est$=$2}
the errors are estimated
based on the eigenvalue decrements history,
which produces an estimate for 
the asymptotic convergence facotr,
the geometrical average of
the eigenvalue error reduction per iteration:
%
\begin{equation}
\label{ssmfe_core:eqn:acf}
q_{ij} =
\left|
\frac
{\lambda_j^i - \lambda_j^{i-1}}
{\lambda_j^i - \lambda_j^0}
\right|^{\frac{1}{i}}
\approx
\left|
\frac
{\lambda_j - \lambda_j^{i-1}}
{\lambda_j - \lambda_j^0}
\right|^{\frac{1}{i}}
=
\left|
\frac
{\lambda_j - \lambda_j^{i-1}}
{\lambda_j - \lambda_j^{i-2}}
\cdots
\frac
{\lambda_j - \lambda_j^1}
{\lambda_j - \lambda_j^0}
\right|^{\frac{1}{i}}
\end{equation}
%
where $\lambda_j^i$ is the approximation to $\lambda_j$ on
$i$-th iteration
(see Technical Report RAL-TR-2010-19 for further details).
%which in turn yields error estimates
%for both eigenvalues and eigenvectors.
Unlike the residual estimates mentioned in this section, 
such `kinematic' error estimates are
not guaranteed to be upper bounds for the actual errors.
However, the numerical tests have demonstrated
that kinematic error estimates 
are significantly more accurate,
i.e. closer to the actual error,
than the residual-based estimates. 
Furthermore, they straightforwardly
apply to the generalized case as well.

\subsubsection{Generalized problems}

In the case
of the generalized eigenvalue problem \EqnRef{ssmfe_core:eqn:evp.g},
all of the residual norms in the previous section must be replaced
with %\linebreak 
$\|\cdot\|_{B^{-1}}$-norm of the residual
$r_j = A \tilde x_j - \tilde\lambda_j B \tilde x_j$
($\|r_j\|_{B^{-1}}^2 = r_j^* B^{-1} r_j$)
or its upper estimate, e.g. 
$\beta_1^{-1/2}\|\cdot\|$,
where $\beta_1$ is the smallest eigenvalue of $B$.
Hence, if $\beta_1$ is known, then
the error tolerances for eigenvalues and eigenvectors
must be multiplied by $\beta_1$ and $\sqrt{\beta_1}$
respectively. If no estimate for $\|\cdot\|_{B^{-1}}$-norm
is available, then the use of
non-zero residual tolerances and
{\tt options\%err\_est$=$1}
is not recommended.
In the case of problem \EqnRef{ssmfe_core:eqn:evp.p},
the residuals are computed as
$r_j = A B \tilde x_j - \tilde \lambda_j \tilde x_j$,
$B$-norms of $r_j$ are used in \EqnRef{ssmfe_core:eqn:aL} and \EqnRef{ssmfe_core:eqn:DK},
and Lehmann matrix becomes
$\hat A = \tilde\Lambda_k - S_k^T B\ S_k$.

\subsection*{References}

[1]
E.~E.~Ovtchinnikov and J.~Reid.
A preconditioned block conjugate gradient
algorithm for computing extreme eigenpairs
of symmetric and Hermitian problems.
Technical Report RAL-TR-2010-19, 2010.

\noindent
[2]
E.~E.~Ovtchinnikov,
{\em Jacobi correction equation, line search and
conjugate gradients in Hermitian eigenvalue computation I:
Computing an extreme eigenvalue},
SIAM J. Numer. Anal., {\bf 46}:2567--2592, 2008.

\noindent
[3]
E.~E.~Ovtchinnikov,
{\em Jacobi correction equation, line search and
conjugate gradients in Hermitian eigenvalue computation II:
Computing several extreme eigenvalues},
SIAM J. Numer. Anal., {\bf 46}:2593--2619, 2008.

\section{Examples}

\subsection{Preconditioning example}

The following code 
computes the 5 leftmost eigenpairs of 
the matrix $A$ of order 100 that approximates 
the two-dimensional Laplacian operator
on a 20-by-20 grid.
One forward and one backward Gauss-Seidel update
are used for preconditioning,
which halves the number of iterations
compared with solving the same problem without preconditioning.
The module {\tt laplace2d} (\texttt{examples/Fortran/ssmfe/laplace2d.f90})
supplies the subroutine {\tt apply\_laplacian()}
that multiplies a block of vectors by $A$,
and a subroutine 
{\tt apply\_gauss\_seidel\_step()}
that computes $y = T x$ for a block of vectors $x$
by applying one forward and one backward update
of the Gauss-Seidel method to the system $A y = x$.
\verbatiminput{examples/Fortran/ssmfe/precond_core.f90}
The code generates the following output:
\begin{verbatim}
  5 eigenpairs converged in  72 iterations
 lambda(1) = 4.4676695E-02
 lambda(2) = 1.1119274E-01
 lambda(3) = 1.1119274E-01
 lambda(4) = 1.7770878E-01
 lambda(5) = 2.2040061E-01
\end{verbatim}
