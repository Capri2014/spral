\packagename{RANDOM}
\version{1.0.0}
\versiondate{7 April 2014}
\purpose{
   This package generates pseudo-random numbers using a linear congruential
   generator. It should generate the same random numbers using any standards
   compliant Fortran compiler on any architecture so long as the default
   integer and real kinds are the same.
   
   The seed can optionally be observed or specified by the user.
}

\title{Pseudo-random Number Generator}
\author{
   Jonathan Hogg (STFC Rutherford Appleton Laboratory) \\
}
\pkglang{Fortran}
\spralmaketitle
\thispagestyle{firststyle}

\section*{Major version history}
\begin{description}
\item[2014-04-07 Version 1.0.0] Initial release
\end{description}

%%%%%%%%%%%%%%%%%%%%%% installation %%%%%%%%%%%%%%%%%%%%%%

\section{Installation}
Please see the SPRAL install documentation.

%%%%%%%%%%%%%%%%%%%%%% how to use %%%%%%%%%%%%%%%%%%%%%%%%

\section{Usage overview}

\subsection{Calling sequences}

Access to the package requires a {\tt USE} statement \\ \\
\indent\hspace{8mm}{\tt use spral\_random} \\

\medskip

\noindent
The following procedures are available to the user:
\begin{itemize}
\item {\tt random\_real()} generates a random number $\mathrm{Unif}(-1,1)$ or
   $\mathrm{Unif}(0,1)$.
\item {\tt random\_integer()} generates a random integer $\mathrm{Unif}(1,\ldots,n)$.
\item {\tt random\_logical()} generates a random boolean value.
\item {\tt random\_get\_seed()} returns the current random seed to the user.
\item {\tt random\_set\_seed()} sets the current random seed.
\end{itemize}

%%%%%%%%%%%%%%%%%%%%%% derived types %%%%%%%%%%%%%%%%%%%%%%%%

\subsection{Derived types} \label{derived types}

The user must employ the derived type {\tt random\_state} defined by the
package to store the state. 
The following pseudo-code illustrates how to declare a scalar of this type.
\begin{verbatim}
      use spral_random
      ...
      type (random_state) :: state
      ...
\end{verbatim}
The components of {\tt random\_state} are not available to the user, but
may be examined and altered through calls to {\tt random\_get\_seed()} and
{\tt random\_set\_seed()} respectively.

\subsection{Optional arguments}\label{Optional arguments}

We use square brackets {\tt [ ]} to indicate {\it optional} arguments.
In each
call, optional arguments appear last in argument list.  Since we
reserve the right to add additional optional arguments in future
releases of the code, {\bf we strongly recommend that all optional
arguments be called by keyword, not by position}.

\subsection{Integer, real and package types}\label{Integer kinds}

{\tt INTEGER} denotes default {\tt INTEGER} and
{\tt INTEGER(long)} denotes {\tt INTEGER(kind=selected\_int\_kind(18))}.

\noindent
{\tt REAL} denotes double precision real.
We also use the term {\bf package type} to mean the same.

%%%%%%%%%%%%%%%%%%%%%% argument lists %%%%%%%%%%%%%%%%%%%%%%%%

\section{Random Generation Subroutines}

%%%%%%%%% analyse subroutine %%%%%%

\subsection{\texttt{random\_real()}}

\textbf{\noindent
   To generate a random number uniformly at random from the interval $(-1,1)$ or $(0,1)$,
   \vspace*{0.1cm} \\
   \texttt{ \hspace*{0.2cm}
      sample = random\_real(state[, positive])
   }
   \vspace{0.3cm}
}

\noindent
The function returns a scalar value {\tt sample} of type {\tt REAL} that is a
sample from $\mathrm{Unif}(0,1)$ if {\tt positive} is present with the value
{\tt .true.} or from $\mathrm{Unif}(-1,1)$ otherwise.

\begin{description}

\item[\texttt{state}] is an \intentinout\ scalar of
type  {\tt random\_state}. It contains the current state of the random number
generator, and need not be initialized by the user.

\item[\texttt{positive}] is an optional \intentin\ scalar of type {\tt LOGICAL}.
   If present with the value {\tt .true.}, the sample will be returned from
   the interval $(0,1)$. Otherwise, the sample will be returned from the
   interval $(-1,1)$.

\end{description}

\subsection{\texttt{random\_integer()}}

\textbf{\noindent
   To generate an integer uniformly at random from the interval $[1,n]$,
   \vspace*{0.1cm} \\
   \texttt{ \hspace*{0.2cm}
      sample = random\_integer(state, n)
   }
   \vspace{0.3cm}
}

\noindent
The function returns a scalar value {\tt sample} of type {\tt INTEGER} that is a
sample from $\mathrm{Unif}(1, \ldots, n)$.

\begin{description}

\item[\texttt{state}] is an \intentinout\ scalar of
type  {\tt random\_state}. It contains the current state of the random number
generator, and need not be initialized by the user.

\item[\texttt{n}] is an optional \intentin\ scalar of type {\tt INTEGER}.
   It specifies the maximum value that sample may take.

\end{description}

\subsection{\texttt{random\_logical()}}

\textbf{\noindent
   To generate a logical from $B(1,0.5)$,
   \vspace*{0.1cm} \\
   \texttt{ \hspace*{0.2cm}
      sample = random\_logical(state)
   }
   \vspace{0.3cm}
}

\noindent
The function returns a scalar value {\tt sample} of type {\tt LOGICAL} that is a
sample from $B(1,0.5)$, i.e. has an equal probability of being {\tt .true.} or
{\tt .false.}.

\begin{description}

\item[\texttt{state}] is an \intentinout\ scalar of
type  {\tt random\_state}. It contains the current state of the random number
generator, and need not be initialized by the user.

\end{description}

\section{Seed control subroutines}

\subsection{\texttt{random\_get\_seed()}}

\textbf{\noindent
   To obtain the current random seed,
   \vspace*{0.1cm} \\
   \texttt{ \hspace*{0.2cm}
      seed = random\_get\_seed(state)
   }
   \vspace{0.3cm}
}

\noindent
The function returns a scalar value {\tt seed} of type {\tt INTEGER} that is the
current seed stored in {\tt state}. The stream of random numbers generated after
this call can be reproduced through the same sequence of calls after {\tt seed}
has been passed to {\tt random\_set\_seed()}.

\begin{description}

\item[\texttt{state}] is an \intentin\ scalar of
type  {\tt random\_state}. It contains the current state of the random number
generator, and need not be initialized by the user.

\end{description}

\subsection{\texttt{random\_set\_seed()}}

\textbf{\noindent
   To set the current random seed,
   \vspace*{0.1cm} \\
   \texttt{ \hspace*{0.2cm}
      call random\_get\_seed(state, seed)
   }
   \vspace{0.3cm}
}

\begin{description}

\item[\texttt{state}] is an \intentinout\ scalar of type {\tt random\_state} to
   be reinitialized with the specified seed.

\item[\texttt{seed}] is an \intentin\ scalar of type {\tt INTEGER} that
   specifies the new seed.

\end{description}

\section{Method} \label{method}

\subsection{PRNG}
We use a linear congruential generator of the following form:
$$
   X_{n+1} = (aX_n + c) \mathrm{mod} m
$$
with the following constants
\begin{eqnarray*}
   a &=& 1103515245, \\
   c &=& 12345, \\
   m &=& 2^{31}.
\end{eqnarray*}
According to Wikipedia, this is the same as used in glibc.

The LCG is evolved before each sample is taken, and the sample is based on the
new value.

The routines {\tt random\_get\_seed()} and {\tt random\_set\_seed()} allow the user to get and set the current value of $X_n$. The default seed, $X_0 = 486502$.

\subsection{\tt random\_real()}
If {\tt positive} is present with value {\tt .true.}, a sample from $\mathrm{Unif}(0,1)$ is generated as
$$
   \frac{\texttt{real}(X_n)}{\texttt{real}(m)},
$$
otherwise, a sample from $\mathrm{Unif}(-1,1)$ is generated as
$$
   1 - \frac{\texttt{real}(2X_n)}{\texttt{real}(m)}.
$$

\subsection{\tt random\_int()}
A random sample from the discrete distribution $\mathrm{Unif}(1, \ldots, n)$ is
generated as
$$
   \texttt{int}\left( X_n \frac{\texttt{real}(n)}{\texttt{real}(m)} \right) + 1.
$$

\subsection{\tt random\_logical()}
A random logical value is generated by evaluating the expression
$$
\tt \left(1\ .eq.\ random\_integer(state, 2)\right).
$$


%%%%%%%%%%%%%%%%%%%%%% EXAMPLE %%%%%%%%%%%%%%%%%%%%%%%%

\section{Example} \label{examples}

The following example code:
\verbatiminput{../examples/Fortran/random.f90}
Produces the following output:
\begin{verbatim}
Some random values
Sample Unif(-1,1)       =   0.951878630556
Sample Unif(0,1)        =   0.395779648796
Sample Unif(1, ..., 20) =                3
Sample B(1,0.5)         =                F

The same random values again
Sample Unif(-1,1)       =   0.951878630556
Sample Unif(0,1)        =   0.395779648796
Sample Unif(1, ..., 20) =                3
Sample B(1,0.5)         =                F
\end{verbatim}
