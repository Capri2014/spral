\packagename{RANDOM}
\version{1.0.0}
\versiondate{7 April 2014}
\purpose{
   This package generates pseudo-random numbers using a linear congruential
   generator. It should generate the same random numbers using any standards
   compliant Fortran compiler on any architecture so long as the default
   integer and real kinds are the same.
   
   The seed can optionally be observed or specified by the user.
}

\title{Pseudo-random Number Generator}
\author{
   Jonathan Hogg (STFC Rutherford Appleton Laboratory) \\
}
\pkglang{C}
\spralmaketitle
\thispagestyle{firststyle}

\section*{Major version history}
\begin{description}
\item[2014-04-07 Version 1.0.0] Initial release
\end{description}

%%%%%%%%%%%%%%%%%%%%%% installation %%%%%%%%%%%%%%%%%%%%%%

\section{Installation}
Please see the SPRAL install documentation.

%%%%%%%%%%%%%%%%%%%%%% how to use %%%%%%%%%%%%%%%%%%%%%%%%

\section{Usage overview}

\subsection{Calling sequences}

Access to the package requires inclusion of either \texttt{spral.h} (for the
entire \spral\ library) or \texttt{spral\_random.h} (for just the relevant routines). i.e.

\begin{verbatim}
   #include "spral.h"
\end{verbatim}

\noindent
The following functions are available to the user:
\begin{itemize}
\item {\tt random\_real()} generates a real uniformally at random from the interval $(-1,1)$ or $(0,1)$.
\item {\tt random\_integer()} generates an integer uniformally at random from the interval $[1,\ldots,n]$.
\item {\tt random\_logical()} generates a random boolean value.
\end{itemize}

\subsection{Seed initialization}
\label{random: seed initialization}
The random number generator's state is stored in the variable {\tt state} that
is common to all calls. Before its first use, {\tt state} should be initialized
by the user to an initial seed value.

For convenience, the preprocessor macro \texttt{SPRAL\_RANDOM\_INITIAL\_SEED}
has been defined, and \texttt{state} may be declared and initialized using a
statement of the following form:
\begin{verbatim}
   int state = SPRAL_RANDOM_INITIAL_SEED;
\end{verbatim}

At any time the user may change the seed, for example to restore a previous
value. The same sequence of calls following the (re-)initialization of seed to
the same value will produce the same pseudo-random values.

%%%%%%%%%%%%%%%%%%%%%% argument lists %%%%%%%%%%%%%%%%%%%%%%%%

\section{Random Generation Subroutines}

%%%%%%%%% analyse subroutine %%%%%%

\subsection{\texttt{random\_real()}}

\textbf{\noindent
   To generate a real uniformly at random from the interval $(-1,1)$ or $(0,1)$, the following routine is provided.
   \vspace*{0.1cm} \\
   \texttt{ \hspace*{0.2cm}
      double spral\_random\_real(int *state, bool positive);
   }
   \vspace{0.3cm}
}

\noindent
The return value is a sample from $\mathrm{Unif}(0,1)$ if {\tt positive} is
{\tt true}, or from $\mathrm{Unif}(-1,1)$ if it is {\tt false}.

\begin{description}

\item[\texttt{*state}] is the current state of the random number generator.
   Before the first call (only) to a package routine, it must be initialized by
   the user as described in Section~\ref{random: seed initialization}.

\item[\texttt{positive}] controls whether the return value is required to be
   positive. If \texttt{positive} is \texttt{true}, the sample will be returned
   from the interval $(0,1)$. Otherwise, the sample will be returned from the
   interval $(-1,1)$.

\end{description}

\subsection{\texttt{random\_integer()}}

\textbf{\noindent
   To generate an integer uniformly at random from the interval $[1,n]$, the
   following routine is provided.
   \vspace*{0.1cm} \\
   \texttt{ \hspace*{0.2cm}
      int spral\_random\_integer(int *state, int n);
   }
   \vspace{0.3cm}
}

\noindent
The return value is a sample from $\mathrm{Unif}(1, \ldots, n)$.

\begin{description}

\item[\texttt{*state}] is the current state of the random number generator.
   Before the first call (only) to a package routine, it must be initialized by
   the user as described in Section~\ref{random: seed initialization}.

\item[\texttt{n}] specifies the upper bound of the range to be sampled.

\end{description}

\subsection{\texttt{random\_logical()}}

\textbf{\noindent
   To generate a logical with equal probability of being \texttt{.true.} or \texttt{.false.}, the following routine is provided.
   \vspace*{0.1cm} \\
   \texttt{ \hspace*{0.2cm}
      bool spral\_random\_logical(int *state);
   }
   \vspace{0.3cm}
}

\noindent
The return value has an equal probability of being {\tt true} or {\tt false}.

\begin{description}

\item[\texttt{*state}] is the current state of the random number generator.
   Before the first call (only) to a package routine, it must be initialized by
   the user as described in Section~\ref{random: seed initialization}.

\end{description}

\section{Method}

\subsection{Pseudo-random number generation}
We use a linear congruential generator (LCG) of the following form:
$$
   X_{n+1} = (aX_n + c)\quad \mathrm{mod}\; m
$$
with the following constants
\begin{eqnarray*}
   a &=& 1103515245, \\
   c &=& 12345, \\
   m &=& 2^{31}.
\end{eqnarray*}
According to Wikipedia, this is the same as used in glibc.

The LCG is evolved before each sample is taken, and the sample is based on the
new value.

The variable \texttt{state} stores the current value of $X_n$, and the macro
\texttt{SPRAL\_RANDOM\_INITIAL\_SEED} specifies the seed $X_0 = 486502$.

\subsection{\tt random\_real()}
If {\tt positive} is present with value {\tt .true.}, a sample from $\mathrm{Unif}(0,1)$ is generated as
$$
   \frac{\texttt{real}(X_n)}{\texttt{real}(m)},
$$
otherwise, a sample from $\mathrm{Unif}(-1,1)$ is generated as
$$
   1 - \frac{\texttt{real}(2X_n)}{\texttt{real}(m)}.
$$

\subsection{\tt random\_int()}
A random sample from the discrete distribution $\mathrm{Unif}(1, \ldots, n)$ is
generated as
$$
   \texttt{int}\left( X_n \frac{\texttt{real}(n)}{\texttt{real}(m)} \right) + 1.
$$

\subsection{\tt random\_logical()}
A random logical value is generated by evaluating the expression
$$
\tt \left(1\ .eq.\ spral\_random\_integer(state, 2)\right).
$$


%%%%%%%%%%%%%%%%%%%%%% EXAMPLE %%%%%%%%%%%%%%%%%%%%%%%%

\section{Example}

The following example code:
\verbatiminput{examples/C/random.c}
Produces the following output:
\begin{verbatim}

Some random values
Sample Unif(-1,1)       =   0.951878630556
Sample Unif(0,1)        =   0.395779648796
Sample Unif(1, ..., 20) =                3
Sample B(1,0.5)         =            false

The same random values again
Sample Unif(-1,1)       =   0.951878630556
Sample Unif(0,1)        =   0.395779648796
Sample Unif(1, ..., 20) =                3
Sample B(1,0.5)         =            false
\end{verbatim}
