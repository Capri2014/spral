\packagename{SSIDS}
\version{1.0.0}
\versiondate{17 March 2014}
\purpose{
   This package solves one or more sets of $n\times n$
   sparse {\bf symmetric} equations  ${AX = B}$ using a multifrontal method on an
   \textbf{NVIDIA GPU}.
   The following cases are covered:
   \begin{enumerate}
   \item $A$ is {\bf indefinite}.
   \texttt{SSIDS} computes the sparse factorization
   $$
      A =  PLD(PL)^T
   $$
   where $P$ is a permutation matrix, $L$ is unit lower triangular,
   and $D$ is block diagonal with blocks of size $1 \times 1 $
   and $2 \times 2$.
   \item $A$ is {\bf positive definite}.
   \texttt{SSIDS} computes the {\bf sparse Cholesky factorization}
   $$
      A =  PL(PL)^T
   $$
   where $P$ is a permutation matrix and $L$ is lower triangular.
   \textit{However, as \texttt{SSIDS} is designed primarily for indefinite
   systems, this may be slower than a dedicated Cholesky solver.}
   \end{enumerate}

   \texttt{SSIDS} returns bit-compatible results.

   An option exists to scale the matrix. In this case, the factorization of
   the scaled matrix  $ {\overline{A} = S A S}$ is computed,
   where ${S}$ is a diagonal scaling matrix.
}

\title{Sparse Symmetric Indefinite Direct Solver}
\author{
   Jonathan Hogg (STFC Rutherford Appleton Laboratory) \\
   Evgueni Ovtchinnikov (STFC Rutherford Appleton Laboratory) \\
   Jennifer Scott (STFC Rutherford Appleton Laboratory)
}
\pkglang{C}
\spralmaketitle
\thispagestyle{firststyle}

\section*{Major version history}
\begin{description}
\item[2014-03-17 Version 1.0.0] Initial release
\end{description}

%%%%%%%%%%%%%%%%%%%%%% installation %%%%%%%%%%%%%%%%%%%%%%

\section{Installation}
Please see the SPRAL install documentation. In particular note that:
\begin{itemize}
   \item A CUDA compiler (\texttt{nvcc}) is required.
   \item A METIS library is required.
   \item A BLAS library is required (in addition to CUBLAS).
   %\item For multi-GPU or multi-stream work, OpenMP must be enabled.
\end{itemize}

%%%%%%%%%%%%%%%%%%%%%% how to use %%%%%%%%%%%%%%%%%%%%%%%%

\section{Usage overview}

\subsection{Calling sequences}

Access to the package requires inclusion of either \texttt{spral.h} (for the
entire \spral library) or \texttt{spral\_ssids.h} (for just the relevant routines). i.e.

\begin{verbatim}
   #include "spral.h"
\end{verbatim}

\noindent
The following functions are available to the user:
\begin{itemize}
\item {\tt spral\_ssids\_default\_options()} initializes the \texttt{options}
structure to default values.

\item {\tt spral\_ssids\_analyse()} accepts the matrix data in compressed
sparse column format and optionally checks it for duplicates and  out-of-range entries.
The user may supply an elimination order; otherwise
one is generated. Using this elimination order,
{\tt spral\_ssids\_analyse()} analyses the sparsity pattern of
the matrix and prepares the data structures for the factorization.

 \item {\tt spral\_ssids\_analyse\_coord()} is an alternative to
{\tt spral\_ssids\_analyse()} that may be used if the user has
the matrix data in coordinate
format. Again, the user may supply an elimination order; otherwise
one is generated. {\tt spral\_ssids\_analyse\_coord()}
checks the matrix data  for duplicates and  out-of-range entries,
stores
it in compressed sparse column format and then proceeds
in the same way as {\tt spral\_ssids\_analyse()}.

\item {\tt spral\_ssids\_factor()} uses the data structures
set up by {\tt spral\_ssids\_analyse()} to compute a sparse
factorization. More than one call to  {\tt spral\_ssids\_factor()}
may follow a call to {\tt spral\_ssids\_analyse()} (allowing more than
one matrix with the same sparsity pattern but different
numerical values to be factorized without multiple calls to
{\tt spral\_ssids\_analyse()}).
An option exists to scale the matrix.

\item {\tt spral\_ssids\_solve1()} and {\tt spral\_ssids\_solve()} use the
   computed factors generated
by  {\tt spral\_ssids\_factor()}
to solve systems ${AX= B}$
for one or more right-hand sides $B$.
Multiple calls to {\tt spral\_ssids\_solve1()} and {\tt spral\_ssids\_solve()}
may follow a call to
{\tt spral\_ssids\_factor()}.
An option is available to perform a partial solution.

\item {\tt spral\_ssids\_free()} should be called after all other calls
are complete for a problem (including after an error
return that does not allow the computation
to continue). It frees memory and resources used by internal data structures.
\end{itemize}

\noindent
In addition, the following routines may be called:
\begin{itemize}
\item {\tt spral\_ssids\_free\_akeep()} and {\tt spral\_ssids\_free\_fkeep()}
may be called to free the memory and resources allocated by 
{\tt spral\_ssids\_analyse() or spral\_ssids\_factorize()} to be freed
separately.
\item {\tt spral\_ssids\_enquire\_posdef()} may be called
in the  positive-definite case to obtain the pivots used.
\item {\tt spral\_ssids\_enquire\_indef()} may be called
in the indefinite case to obtain the pivot sequence used by the factorization
and the entries of  ${D}^{-1}$.
\item {\tt spral\_ssids\_alter()} may be called in the indefinite case to alter
the entries of ${D}^{-1}$.
Note that this means that  $PLD(PL)^T$ is no longer
a factorization of $A$.

\end{itemize}

%%%%%%%%%%%%%%%%%%%%%% derived types %%%%%%%%%%%%%%%%%%%%%%%%

\subsection{Derived types} \label{ssids: derived types}

For each problem, the user must employ the derived types defined by the
package to declare scalars of the types {\tt struct spral\_ssids\_options} and
{\tt struct spral\_ssids\_inform}. The user must also declare two
\texttt{void~*} pointers \texttt{akeep} and \texttt{fkeep} for the package's
private data structures. The options data structure \textbf{must} be initialized
using \texttt{spral\_ssids\_default\_options()}, and the pointers \texttt{akeep}
and \texttt{fkeep} must be initialized to \texttt{NULL}.
The following pseudo-code illustrates this.
\begin{verbatim}
      #include "spral.h"
      ...
      struct spral_ssids_options options;
      struct spral_ssids_inform inform;
      void *akeep = NULL;
      void *fkeep = NULL;
      ...
      spral_ssids_default_options(&options);
      ...
\end{verbatim}
The components of {\tt spral\_ssids\_options} and {\tt spral\_ssids\_inform} are explained
in Sections~\ref{ssids: typeoptions} and \ref{ssids: typeinform}.
The \texttt{void~*} pointers are allocated using Fortran, and must be passed to
\texttt{spral\_ssids\_free\_akeep()}, \texttt{spral\_ssids\_free\_fkeep()} and/or
\texttt{spral\_ssids\_free()} to free the associated memory.

\subsection{Achieving bit-compatibility} \label{ssids: bitcompat}
Care has been taken to ensure bit-compatibility is achieved using this solver.
That is, consecutive runs with the same data on the same machine  produces exactly
the same solution.

\subsection{Data formats} \label{ssids: dataformats}

\begin{figure}
   \caption{ \label{ssids: format eg}
      Data format example matrix
   }
   $$
      \left( \begin{array}{ccccc}
         1.1 & 2.2 &     & 3.3 &     \\
         2.2 &     & 4.4 &     &     \\
             & 4.4 & 5.5 &     & 6.6 \\
         3.3 &     &     & 7.7 & 8.8 \\
             &     & 6.6 & 8.8 & 9.9
      \end{array} \right)
   $$
\end{figure}

\subsubsection{Compressed Sparse Column (CSC) Format} \label{ssids: cscformat}
This standard data format consists of the following data:
\begin{verbatim}
   int    n;                  /* size of matrix */
   int    ptr[n+1];           /* column pointers */
   int    row[ ptr[n]-1 ];    /* row indices */
   double val[ ptr[n]-1 ];    /* numerical values */
\end{verbatim}
Non-zero matrix entries are ordered by increasing column index and stored in
the arrays \texttt{row[]} and \texttt{val[]} such that \texttt{row[k]} holds
the row number and \texttt{val[k]} holds the value of the \texttt{k}-th entry.
The \texttt{ptr[]} array stores column pointers such that \texttt{ptr[i]} is
the position in \texttt{row[]} and \texttt{val[]} of
the first entry in the \texttt{i}-th column, and \texttt{ptr[n]} is
the total number of entries.
Entries that are zero, including those on the diagonal, need not be specified.

If this format is used, \texttt{SSIDS} requires only the lower triangular entries of $A$, and there 
should be no duplicate entries. If the \texttt{check}
argument to
\texttt{ssids\_analyse()} is \texttt{true}, out-of-range entries (including
those in the upper triangle) will be discarded and any duplicates will be
summed.

To illustrate the CSC format, the following arrays describe the matrix shown in
Figure~\ref{ssids: format eg}.
\begin{verbatim}
   int n = 5;
   int ptr[]    = { 1,             4,   5,        7,        9,    10 };
   int row[]    = { 1,   2,   4,   3,   3,   5,   4,   5,   5 };
   double val[] = { 1.1, 2.2, 3.3, 4.4, 5.5, 6.6, 7.7, 8.8, 9.9 };
\end{verbatim}

\subsubsection{Coordinate Format} \label{ssids: coordformat}
This standard data format consists of the following data:
\begin{verbatim}
   int    n;         /* size of matrix */
   int    ne;        /* number of non-zero entries */
   int    row[ne];   /* row indices */
   int    col[ne];   /* column indices */
   double val[ne];   /* numerical values */
\end{verbatim}
The arrays should be set such that the \texttt{k}-th entry is in row
\texttt{row[k]} and column \texttt{col[k]} with value \texttt{val[k]}.
Entries that are zero, including those on the diagonal, need not be specified.

If this format is used,
\texttt{SSIDS} requires that each entry of $A$ should be present \textbf{only} in the
lower \textit{or} upper triangular part. Entries present in both will be summed, as
will any duplicate entries. Out-of-range entries are ignored.

To illustrate the coordinate format, the following arrays describe the matrix shown in
Figure~\ref{ssids: format eg}.
\begin{verbatim}
   int n = 5;
   int ne = 9;
   int row[]    = { 1,   2,   3,   4,   3,   5,   4,   5,   5 };
   int col[]    = { 1,   1,   2,   1,   3,   3,   4,   4,   5 };
   double val[] = { 1.1, 2.2, 4.4, 3.3, 5.5, 6.6, 7.7, 8.8, 9.9 };
\end{verbatim}

%%%%%%%%%%%%%%%%%%%%%% argument lists %%%%%%%%%%%%%%%%%%%%%%%%

\section{Basic Subroutines}

%%%%%%%%% options initialization subroutine %%%%%%
\subsection{\texttt{spral\_ssids\_default\_options()}}

\textbf{To initialize a variable of type \texttt{struct spral\_ssids\_options},
   the following routine is provided.}

\vspace*{0.1cm}
\noindent
\textbf{\texttt{
      \hspace*{0.3cm} void spral\_ssids\_default\_options(struct spral\_ssids\_options *options);
}}

\noindent
\begin{description}
   \item[\texttt{*options}] is the instance to be initialized.
\end{description}

%%%%%%%%% analyse subroutine %%%%%%

\subsection{\texttt{spral\_ssids\_analyse()} and \texttt{spral\_ssids\_analyse\_coord()}}\label{ssids: analyse}

\textbf{
   To analyse the sparsity pattern and prepare for the factorization,
   the following routines are provided.
   \begin{itemize}
      \item[] for Compressed Sparse Column (CSC) format:
      \vspace*{0.1cm} \\
      \texttt{
         \hspace*{0.2cm} void spral\_ssids\_analyse(bool check, int n, int *order, const int *ptr, \\
         \hspace*{0.7cm} const int *row, const double *val, void **akeep, \\
         \hspace*{0.7cm} const struct spral\_ssids\_options *options,
         struct spral\_ssids\_inform *inform);
      }
      \item[] for Coordinate format:
      \vspace*{0.1cm} \\
      \texttt{
         \hspace*{0.2cm} void spral\_ssids\_analyse\_coord(int n, int *order, int ne, const int *row, \\
         \hspace*{0.7cm} const int *col, const double *val, void **akeep, \\
         \hspace*{0.7cm} const struct spral\_ssids\_options *options,
            struct spral\_ssids\_inform *inform);
      }
   \end{itemize}
}

\noindent
Matrix data should be supplied as described in Section~\ref{ssids: dataformats}. As
the package uses CSC format internally, \texttt{spral\_ssids\_analyse()} with checking
disabled provides the most efficient interface.

\noindent
\begin{description}

\item[\texttt{check}] determines if the matrix is checked for errors
   (\texttt{true}) or not (\texttt{false}). If set to {\tt true} the matrix
   data is checked for errors and the cleaned matrix (duplicates are summed and
   out-of-range entries discarded) is stored in {\tt akeep}.
   Otherwise, for data in CSC format, no checking of the matrix data is carried
   out and {\tt ptr[]} and {\tt row[]} must be passed unchanged to the
   factorization routines. 
   Checking is always performed when the coordinate format is used.

\item[\texttt{n}{\rm ,} \texttt{ptr[n+1]}{\rm ,} \texttt{row[ptr[n]]}] specify the lower
   triangular part of $A$ in CSC format (see Section~\ref{ssids: cscformat}).

\item[\texttt{n}{\rm ,} \texttt{ne}{\rm ,} \texttt{row[ne]}{\rm ,} \texttt{col[ne]}]
   specify the lower (or upper) triangular part of $A$ in coordinate format
   (see Section~\ref{ssids: coordformat}).

\item[\texttt{order[n]}] may be \texttt{NULL}. If {\tt options.ordering}$=${\tt 0},
   {\tt order(:)} must non-\texttt{NULL} and {\tt order[i]} must hold the
   position of variable $i$ in the elimination order. On exit, {\tt order[]}
   contains the elimination order that {\tt spral\_ssids\_factor()} will be given (it
   is passed to these routines as part of {\tt *akeep}); this order may give
   slightly more fill-in than any user-supplied order and, in the indefinite
   case, may be modified by {\tt spral\_ssids\_factor()} to maintain numerical
   stability. 

\item[\texttt{val[ne]}] may be \texttt{NULL}, if non-\texttt{NULL} it must hold
   the numerical values of the entries of the matrix, as described in
   Section~\ref{ssids: dataformats}.
   {\tt val[]} must not be \texttt{NULL} if a matching-based elimination
   ordering is required ({\tt options.ordering$=$2}), and is otherwise ignored.

\item[\texttt{*akeep}] is allocated to hold data about the problem being
   solved and must be passed unchanged to the other subroutines.

\item[\texttt{*options}] specifies the algorithmic options used by the
   routine, as explained in Section~\ref{ssids: typeoptions}.

\item[\texttt{*inform}] is used to return information about the execution
   of the routine, as explained in Section~\ref{ssids: typeinform}.

\end{description}


%%%%%%%%% factorize subroutine %%%%%%
\subsection{\texttt{spral\_ssids\_factor()}} \label{ssids: factorize}
\textbf{To factorize the matrix, the following routine is provided.
   \vspace*{0.1cm} \\
   \texttt{
      \hspace*{0.2cm} void spral\_ssids\_factor(bool posdef, const int *ptr, const int *row, const double *val, \\
      \hspace*{0.7cm} double *scale, void *akeep, void **fkeep,
         const struct spral\_ssids\_options *options, \\
      \hspace*{0.7cm} struct spral\_ssids\_inform *inform);
   }
}

\begin{description}
\item[\texttt{posdef}] must be {\tt true} if the matrix is positive-definite,
   and {\tt false} if it is indefinite.

\item[\texttt{ptr[n+1]}] and {\tt row[ptr[n]]} may both be \texttt{NULL} unless 
   {\tt spral\_ssids\_analyse()} was called with {\tt check} set to {\tt false}.
   In this case, they must both be non-\texttt{NULL} and must be unchanged
   since that call.

\item[\texttt{val[ptr[n]]}] must hold the numerical values of the entries of the
   matrix, as described in Section~\ref{ssids: dataformats}.

\item[\texttt{scale[n]}] may be \texttt{NULL}. If non-\texttt{NULL} and
   \texttt{options.scaling}$=$\texttt{0},
   it must contain the diagonal entries of the scaling matrix ${S}$.
   On exit, \texttt{scale[i]} will contain the \texttt{i}-th
   diagonal entry of the scaling matrix $S$.

\item[\texttt{akeep}] must be unchanged since the call to
   {\tt spral\_ssids\_analyse()} or {\tt spral\_ssids\_analyse\_coord()}.

\item[\texttt{*fkeep}] is allocated to hold data about the problem being
   solved and must be passed unchanged to the other subroutines.

\item[\texttt{*options}] specifies the algorithmic options used by the
   subroutine, as explained in Section~\ref{ssids: typeoptions}.

\item[\texttt{*inform}] is used to return information about the execution
   of the subroutine, as explained in Section~\ref{ssids: typeinform}.

\end{description}

%%%%%%%%% solve subroutine %%%%%%
\subsection{\texttt{spral\_ssids\_solve()}} \label{ssids: solve}
\textbf{ To solve the linear system $AX=B$, after a call to
   \texttt{spral\_ssids\_factor()}, the following routines are provided.
   \begin{itemize}
      \item[] for a single right-hand side:
      \vspace*{0.1cm} \\
      \texttt{
         \hspace*{0.2cm} void spral\_ssids\_solve1(int job, double *x1, void *akeep, void *fkeep, \\
         \hspace*{0.7cm} const struct spral\_ssids\_options *options,
            struct spral\_ssids\_inform *inform);
      }
      \item[] for one or more right-hand sides:
      \vspace*{0.1cm} \\
      \texttt{
         \hspace*{0.2cm} void spral\_ssids\_solve(int job, int nrhs, double *x, int ldx, void *akeep, \\
         \hspace*{0.7cm} void *fkeep, const struct spral\_ssids\_options *options, \\
         \hspace*{0.7cm} struct spral\_ssids\_inform *inform);
      }
   \end{itemize}
}

\begin{description}
\item[\texttt{job}] is used to specify the solve to be performed.
   In the positive-definite case, the Cholesky factorization that has been
   computed may be expressed in the form
   \[ {SAS} = ({PL})({PL})^T \]
   where $P$ is a permutation matrix and $L$ is lower triangular.
   In the indefinite case, the factorization that has been computed may be
   expressed in the form
   \[ {S AS} = ({PL}){D}({PL})^T \]
   where $P$ is a permutation matrix, $L$ is unit lower triangular, and $D$ is
   block diagonal with blocks of order 1 and 2. $S$ is a diagonal scaling
   matrix ($S$ is equal to the identity, if \texttt{options.scaling=0} and
   {\tt scale} is \texttt{NULL} on the last call to {\tt spral\_ssids\_factor}()).
   A partial solution may be computed by setting {\tt job} to have one of the
   following values:
   \begin{description}
   \item[\texttt{0}] for solving $AX = B$
   \item[\texttt{1}] for solving ${PLX} = {SB}$
   \item[\texttt{2}] for solving ${DX} = {B}$ (indefinite case only)
   \item[\texttt{3}] for solving $({PL})^T{S^{-1}X} = {B}$
   \item[\texttt{4}] for solving $D({PL})^T{S^{-1}X} = {B}$ (indefinite case only)
   \end{description}

\item[\texttt{x1[n]}] must be set
   such that \texttt{x1[i]} holds the component of the right-hand side for
   variable $i$. On exit, \texttt{x1[i]} holds the solution for variable
   $i$.

\item[\texttt{nrhs}] holds the number of right-hand sides.

\item[\texttt{x[nrhs][ldx]}] must be set so that \texttt{x[j][i]} holds the
   component of the right-hand side for variable $i$ to the $j$-th
   system. On exit, \texttt{x[j][i]} holds the solution for variable $i$ to
   the $j$-th system.

\item[\texttt{ldx}] must be set to the leading dimension of array \texttt{x[]}.

\item[\texttt{akeep}] must be unchanged since the last call to
   {\tt spral\_ssids\_factor()}.

\item[\texttt{fkeep}] must be unchanged since the last call to
   {\tt spral\_ssids\_factor()}.

\item[\texttt{*options}] specifies the algorithmic options used by the
   subroutine, as explained in Section~\ref{ssids: typeoptions}.

\item[\texttt{inform}] is used to return information about the execution
   of the subroutine, as explained in Section~\ref{ssids: typeinform}.

\end{description}

%%%%%%% termination subroutine %%%%%%

\subsection{\texttt{spral\_ssids\_free()}, \texttt{spral\_ssids\_free\_akeep()}, and \texttt{spral\_ssids\_free\_fkeep()}}
Once all other calls are complete for a problem or after an error
return that does not allow the computation to continue,
a call should be made to free memory and CUDA resources allocated by
\texttt{SSIDS} and associated with the derived data types {\tt akeep} and/or
{\tt fkeep} using calls to \texttt{spral\_ssids\_free()}.

\vspace{0.1cm}
\noindent
\textbf{To free memory and resources, the following routines are provided.
   \begin{itemize}
      \item[] allocated by \texttt{spral\_ssids\_analyse()} or \texttt{spral\_ssids\_analyse\_coord()}:
      \vspace*{0.1cm} \\
      \texttt{ \hspace*{0.2cm}
         int spral\_ssids\_free\_akeep(void **akeep);
      }
      \item[] allocated by \texttt{spral\_ssids\_factor()}:
      \vspace*{0.1cm} \\
      \texttt{ \hspace*{0.2cm}
         int spral\_ssids\_free\_fkeep(void **fkeep);
      }
      \item[] both at once:
      \vspace*{0.1cm} \\
      \texttt{ \hspace*{0.2cm}
         int spral\_ssids\_free(void **akeep, void **fkeep);
      }
   \end{itemize}
}

\noindent
A non-zero return value gives a CUDA error code. This may indicate either
a failure to deallocate GPU memory, or a pre-existing CUDA error condition.
Note that due to the asynchronous nature of GPU execution, the
reported error may have a cause errors external to {\tt SSIDS}.

\begin{description}

\item[\texttt{*akeep}] must be passed unchanged by the user.
On exit, associated memory and CUDA resources will have been released, and
\texttt{*akeep} set to \texttt{NULL}.

\item[\texttt{*fkeep}] that must be passed unchanged by the user.
On exit, associated memory and CUDA resources will have been released, and
\texttt{*fkeep} set to \texttt{NULL}.

\end{description}

\section{Advanced subroutines}

%%%%%%% enquire subroutine %%%%%%

\subsection{\texttt{spral\_ssids\_enquire\_posdef()}}
\textbf{To obtain the matrix $D$ following a positive-definite factorization, the following routine is provided.
   \vspace*{0.1cm} \\
   \texttt{
      \hspace*{0.2cm} void spral\_ssids\_enquire\_posdef(const void *akeep, const void *fkeep, \\
      \hspace*{0.7cm} const struct spral\_ssids\_options *options,
         struct spral\_ssids\_inform *inform, \\
      \hspace*{0.7cm} double *d);
   }
}

\begin{description}

\item[\texttt{akeep}] must be unchanged since the call to
   {\tt spral\_ssids\_factor()}.

\item[\texttt{fkeep}] must be unchanged since the call to
   {\tt spral\_ssids\_factor()}.

\item[\texttt{*options}] specifies the algorithmic options used by the
   subroutine, as explained in Section~\ref{ssids: typeoptions}.

\item[\texttt{*inform}] is used to return information about the execution
   of the subroutine, as explained in Section~\ref{ssids: typeinform}.

\item[\texttt{d[n]}] holds, on exit, the diagonal entries of the matrix $D$.

\end{description}
%%%%%%% enquire subroutine %%%%%%

\subsection{\texttt{spral\_ssids\_enquire\_indef()}}
\textbf{To obtain the matrix $D^{-1}$ and/or the pivot order following an
   indefinite factorization, the following routine is provided.
   \vspace*{0.1cm} \\
   \texttt{
      \hspace*{0.2cm} void spral\_ssids\_enquire\_indef(const void *akeep, const void *fkeep, \\
      \hspace*{0.7cm} const struct spral\_ssids\_options *options,
         struct spral\_ssids\_inform *inform, \\
      \hspace*{0.7cm} int *piv\_order, double *d);
   }
}

\begin{description}

\item[\texttt{akeep}] must be unchanged since the call to
   {\tt spral\_ssids\_factor()}.

\item[\texttt{fkeep}] must be unchanged since the call to
   {\tt spral\_ssids\_factor()}.

\item[\texttt{*options}] specifies the algorithmic options used by the
   subroutine, as explained in Section~\ref{ssids: typeoptions}.

\item[\texttt{*inform}] is used to return information about the execution
   of the subroutine, as explained in Section~\ref{ssids: typeinform}.

\item[\texttt{piv\_order[n]}] may be \texttt{NULL}. If non-\texttt{NULL}
   then, on exit, \texttt{piv\_order[i]}
   holds the position of variable $i$ in the pivot order.

\item[\texttt{d[n][2]}] may be \texttt{NULL}.
   If non-\texttt{NULL}, on exit diagonal entries of ${D}^{-1}$ will be placed
   in {\tt d[i][0]}, $i = 1,2,\ldots,n$, the off-diagonal entries of ${D}^{-1}$
   will be placed in {\tt d[i][1]}, $i = 1,2,\ldots,n-1$, and {\tt d[n][1]}
   will be set to zero.

\end{description}


%%%%%%% alter subroutine %%%%%%

\subsection{\texttt{spral\_ssids\_alter()}}
\textbf{To alter ${\bf D}^{-1}$ following an indefinite factorization,
   the following routine is provided.
   \vspace*{0.1cm} \\
   \texttt{
      \hspace*{0.2cm} void spral\_ssids\_alter(const double *d, const void *akeep, void *fkeep, \\
      \hspace*{0.7cm} const struct spral\_ssids\_options *options,
         struct spral\_ssids\_inform *inform);
   }
}

\vspace{0.3cm}
\noindent
Note that this routine is not compatabile with the option \texttt{options.presolve}$=$\texttt{1}.

\begin{description}

\item[\texttt{d[n][2]}] must hold the new entries of the matrix $D$.
   The diagonal entries of ${D}^{-1}$ will be altered to
   {\tt d[i][0]}, $i = 1,2,\ldots,n$, and the off-diagonal entries will be
   altered to {\tt d[i][1]}, $i = 1,2,\ldots,n-1$ (and $PLD(PL)^T$ will no
   longer be a factorization of $A$).

\item[\texttt{akeep}] must be unchanged since the call to
   {\tt spral\_ssids\_factor()}.

\item[\texttt{fkeep}] must be unchanged since the call to
   {\tt spral\_ssids\_factor()}.

\item[\texttt{*options}] specifies the algorithmic options used by the
   subroutine, as explained in Section~\ref{ssids: typeoptions}.

\item[\texttt{*inform}] is used to return information about the execution
   of the subroutine, as explained in Section~\ref{ssids: typeinform}.

\end{description}

%%%%%%%%%%% options type %%%%%%%%%%%

\section{Derived types}
\subsection{\texttt{struct spral\_ssids\_options}}
\label{ssids: typeoptions}

The structure {\tt spral\_ssids\_options} is used to specify the options used
within \texttt{SSIDS}. The components, that must be given default values through
a call to \texttt{spral\_ssids\_default\_options()}, are: \\

%%%%%%%%%%%%

\subsubsection*{C specific options}
\begin{description}
\item[\texttt{int array\_base}] specifies the array indexing base. It must
   have the value either \texttt{0} (C indexing) or \texttt{1} (Fortran
   indexing). If \texttt{array\_base}$=$\texttt{1}, the entries of arrays
   \texttt{ptr[]}, \texttt{row[]}, \texttt{col[]}, \texttt{order[]}, and
   \texttt{piv\_order[]} start at 1, not 0. Further, entries of
   \texttt{piv\_order[]} may also have negative sign to indicate they are
   part of a $2\times2$ pivot.
   The default value is \texttt{array\_base}$=$\texttt{0}.
\end{description}

\subsubsection*{Printing options}

\begin{description}

\item[\texttt{int print\_level}]  is used to control the level of printing.
   The different levels are:
\begin{description}
\item{\tt $<$ 0 } No printing.
\item{\tt = 0 } Error and warning messages only.
\item{\tt = 1 } As 0, plus basic diagnostic printing.
\item{\tt $>$ 1 } As 1, plus some additional diagnostic printing.
\end{description}
The default is {\tt print\_level$=$\tt 0}.

\item[\texttt{int unit\_diagnostics}]  holds Fortran the
unit number for diagnostic printing. Printing is suppressed if
{\tt unit\_diagnostics$<0$}.
The default is {\tt unit\_diagnostics$=$6}.

\item[\texttt{int unit\_error}] holds the Fortran
unit number for error messages.
Printing of error messages
is suppressed if {\tt unit\_error$<$0}.
The default is {\tt unit\_error$=$6}.

\item[\texttt{int unit\_warning}] holds the Fortran
unit number for warning messages.
Printing of warning messages is suppressed if {\tt unit\_warning$<$0}.
The default is {\tt unit\_warning$=$6}.

\end{description}



%%%%%%%%%%%%
\subsubsection*{Options used by {\tt spral\_ssids\_analyse()} and
{\tt spral\_ssids\_analyse\_coord()}}

\begin{description}

\item[\texttt{int ordering}] controls the ordering algorithm used. 
If set to {\tt 0}, the user
must supply an elimination order in {\tt order[]}; otherwise
{\tt ssids\_analyse()} or {\tt ssids\_analyse\_coord()} will an compute an
elimination order.
The options are:
\begin{description}
\item{} {\tt 0} User-supplied ordering is used.
\item{} {\tt 1} METIS ordering with default settings is used.
\item{} {\tt 2} A matching-based elimination ordering is computed (the Hungarian
algorithm is used to identify large off-diagonal entries. A restricted METIS
ordering is then used that forces these on to the subdiagonal).
This option should only be chosen for indefinite systems.
A scaling is also computed that may be used in {\tt spral\_ssids\_factor()}
(see {\tt options.scaling} below).
\end{description}
The default is {\tt ordering}$=${\tt 1}.
{\bf Restriction:} {\tt ordering}$=${\tt 0}, {\tt 1}, {\tt 2}.

\item[\texttt{int nemin}] controls
node amalgamation. Two neighbours in the elimination tree are merged
if they both involve fewer than {\tt nemin} eliminations.
The default is {\tt nemin$=$8}.
The default is used if {\tt nemin$<$1}.
\end{description}

%%%%%%%%%%%%
\subsubsection*{Options used by {\tt spral\_ssids\_factor()}}
\begin{description}
\item[\texttt{int scaling}] controls
the use of scaling. The available
options are:
\begin{description}
   \item[\texttt{ $\le$ 0 }] No scaling (if \texttt{scale} is \texttt{NULL}),
      or user-supplied scaling (if \texttt{scale} is non-\texttt{NULL}).
   \item[\texttt{ $=$ 1 }] Compute a scaling using a weighted bipartite matching
      via the Hungarian Algorithm (\texttt{MC64} algorithm).
   \item[\texttt{ $=$ 2 }] Compute a scaling using a weighted bipartite matching
      via the Auction Algorithm (may be lower quality than that computed using 
      the Hungarian Algorithm, but can be considerably faster).
   \item[\texttt{ $=$ 3 }] A matching-based ordering has been generated during the
      analyse phase using {\tt options.ordering $=$ 2}. Use the
      scaling generated as a side-effect of this process. The scaling will be
      the same as that generated with {\tt options.scaling $=$ 1} if the matrix
      values have not changed. This option will generate an error if a
      matching-based ordering was not used.
   \item[\texttt{ $\ge$ 4 }] Compute a scaling using the norm-equilibration
      algorithm of Ruiz.
\end{description}
The default is {\tt scaling}$=${\tt 0}.

\end{description}

%%%%%%%%%%%%
\subsubsection*{Options used by {\tt spral\_ssids\_factor()} with
{\tt posdef}$ =${\tt false}  ($A$ indefinite)}

\begin{description}
\item[\texttt{bool action}] controls the behaviour for singular matrices.
If the matrix is found to be singular (has rank less than the number of
non-empty rows), the computation continues after issuing a warning if
{\tt action} has the value {\tt true} or
terminates with an error if it has the value {\tt false}.
The default is {\tt action}$=${\tt true}.

%\item[\texttt{small}] is a scalar of type {\tt REAL}.
%Any pivot whose modulus is less than {\tt small} is treated as zero.
%The default is {\tt small}$ = {\tt 10^{-20}}$.

\item[double \texttt{u}] holds the relative pivot
tolerance $u$.
The default is {\tt u}$=${\tt 0.01}.
Values outside the range $[0,0.5]$ are treated as the closest value in that range.

\end{description}

%%%%%%%%%%%%
\subsubsection*{Options used by {\tt spral\_ssids\_factor()} and {\tt spral\_ssids\_solve()}}

\begin{description}
\item[\texttt{bool use\_gpu\_solve}] controls
   whether to use the CPU or GPU for the \texttt{spral\_ssids\_solve()}. If
   \texttt{true}, the GPU is used, but some more advanced features are not
   available (see the description of the \texttt{job} parameter in
   Section~\ref{ssids: solve}).
   Setting \texttt{use\_gpu\_solve}$=${\tt false} is only compatible with
   \texttt{options.presolve}$=${\tt 0}.
   The default value is \texttt{use\_gpu\_solve}$=$\texttt{true}.
\item[\texttt{int presolve}] controls the
   amount of extra work performed during the factorization to accelerate the solve.
   It can take the following values:
   \begin{description}
      \item[\texttt{0}] Minimal work is performed during {\tt spral\_ssids\_factor()}
         to prepare for the solve.
      \item[\texttt{1}] The explicit inverse of the
         \texttt{nelim}$\times$\texttt{nelim} diagonal block in each supernode is
         precalculated during {\tt spral\_ssids\_factor()} (where \texttt{nelim} is
         the number of variables eliminated at that supernode). As the matrix
         $L$ is overwritte, the routine {\tt spral\_ssids\_alter()} cannot be
         used.
         This option is not compatible with {\tt options.use\_gpu\_solve}$=$\texttt{false}.
   \end{description}
   The default option is \texttt{presolve}$=$\texttt{0}.
\end{description}

%%%%%%%%%%% inform type %%%%%%%%%%%

\subsection{\texttt{struct spral\_ssids\_inform}}
\label{ssids: typeinform}
The structure {\tt spral\_ssids\_inform}
is used to hold parameters that give information about the progress and needs
of the algorithm. The components of {\tt struct spral\_ssids\_inform}
(in alphabetical order) are:

\begin{description}

\item[\texttt{ing flag}] gives the exit status of the algorithm (details in Section \ref{ssids: errors}).

\item[\texttt{int matrix\_dup}] holds, on exit from
{\tt spral\_ssids\_analyse()} with {\tt check} set to {\tt true} or on exit from
{\tt spral\_ssids\_analyse\_coord()}, the
number of duplicate entries that were found and summed.

\item[\texttt{int matrix\_missing\_diag}] holds, on exit from
{\tt spral\_ssids\_analyse()} with {\tt check} set to {\tt true},
or exit from {\tt ssids\_analyse\_coord()}, the number of diagonal
entries without an explicitly provided value.

\item[\texttt{int matrix\_outrange}] holds, on exit from
{\tt spral\_ssids\_analyse()} with {\tt check} set to {\tt true} or exit from 
{\tt spral\_ssids\_analyse\_coord()}, the
number of out-of-range entries that were   found and discarded.

\item[\texttt{int matrix\_rank}] holds, on exit from
{\tt spral\_ssids\_analyse()} or on exit from {\tt spral\_ssids\_analyse\_coord()},
the structural rank of $A$, if available (otherwise, it is set to {\tt n}).
On exit from
{\tt spral\_ssids\_factor()}, it holds the computed rank of
the factorized matrix.

\item[\texttt{int maxdepth}] holds, on exit from
{\tt spral\_ssids\_analyse()} or {\tt spral\_ssids\_analyse\_coord()},
the maximum depth of the assembly tree.

\item[\texttt{int maxfront}] holds, on exit from
{\tt spral\_ssids\_analyse()} or {\tt spral\_ssids\_analyse\_coord()}, 
the maximum front size
in the positive-definite case (or in the indefinite case with
the same pivot sequence). On exit from
{\tt spral\_ssids\_factor()}, it holds the maximum front size.

\item[\texttt{int num\_delay}] holds, on exit from
{\tt spral\_ssids\_factor()}, the
number of eliminations that were
delayed, that is, the total number of fully-summed
variables that were passed to the father node because
of stability considerations. If a variable is passed
further up the tree, it will be counted again.

\item[\texttt{long num\_factor}] holds, on exit from
{\tt spral\_ssids\_analyse()}  or {\tt spral\_ssids\_analyse\_coord()},
the number of entries that will be in the
factor $L$ in the positive-definite case (or in the indefinite case with
the same pivot sequence). On exit from
{\tt ssids\_factor()},
it holds the actual number of entries in the factor
$L$. In the indefinite case, {\tt 2n} entries of ${D}^{-1}$ are
also held.

\item[\texttt{long num\_flops}] holds, on exit from
{\tt spral\_ssids\_analyse()}  or {\tt spral\_ssids\_analyse\_coord()},
the number of
floating-point operations that
will be needed to perform the factorization
in the positive-definite case (or in the indefinite case  with
the same pivot sequence).  On exit from {\tt spral\_ssids\_factor()}, it holds the
number of floating-point operations performed.

\item[\texttt{int num\_neg}] holds, on exit from {\tt spral\_ssids\_factor()},
the number of negative eigenvalues of the matrix $D$.


\item[\texttt{int num\_sup}] holds, on exit from {\tt spral\_ssids\_analyse()}  or {\tt spral\_ssids\_analyse\_coord()},
the number of supernodes in the problem.

\item[\texttt{int num\_two}] holds, on exit from
{\tt spral\_ssids\_factor()}, the number
of $2 \times 2$ pivots used by the factorization, that is,
the number of $2 \times 2$ blocks in $D$.

\item[\texttt{int stat}] holds, in the event of an allocation or deallocation
error, the Fortran {\tt stat} parameter if it is available
(and is set to {\tt 0} otherwise).

\item[\texttt{int cublas\_error}] holds, in the event of an error return from
the CUBLAS library, the error code returned (and is {\tt 0} otherwise).

\item[\texttt{int cuda\_error}] holds, in the event of a CUDA error, the error code
returned (and is {\tt 0} otherwise). Note that due to the asynchronous nature
of GPU execution, the reported error may have a cause errors external to
{\tt SSIDS}.

\end{description}


%%%%%%%%%%%%%%%%%%%%%% Warning and error messages %%%%%%%%%%%%%%%%%%%%%%%%

\section{Error flags} \label{ssids: errors}

A successful return from a routine in the package is indicated by
{\tt inform.flag} having the value zero.
A negative value is associated with an error message that by default will
be output on unit {\tt options.unit\_error}.

Possible negative values are:

\begin{description}
\item{$-$1} An error has been made in the sequence of calls (this includes
            calling a subroutine after an error that cannot be recovered from).
\item{$-$2} Returned by {\tt spral\_ssids\_analyse()} and {\tt spral\_ssids\_analyse\_coord()}
            if {\tt n$<$0}. Also returned by {\tt spral\_ssids\_analyse\_coord()} if
            {\tt ne$<$1}.
\item{$-$3} Returned by {\tt spral\_ssids\_analyse()} if there is an error in
            {\tt ptr[]}.
\item{$-$4} Returned by {\tt spral\_ssids\_analyse()} if all the variable indices in
            one or more columns  are out-of-range. Also returned by
            {\tt spral\_ssids\_analyse\_coord()} if all entries are out-of-range.
\item{$-$5} Returned by {\tt spral\_ssids\_factor()} if
            {\tt posdef}$=${\tt false} and
            {\tt options.action = false} when the matrix is found to be
            singular. The user may reset the matrix values in {\tt val[]}
            and recall {\tt spral\_ssids\_factor()}.
\item{$-$6} Returned by {\tt spral\_ssids\_factor()} if
            {\tt posdef}$=${\tt true} and the matrix is found to
            be not positive definite. This may be because the Hungarian scaling
            determined that the matrix was structurally singular. The user may
            reset the matrix values in {\tt val[]} and recall
            {\tt spral\_ssids\_factor()}.
\item{$-$7} Returned by {\tt spral\_ssids\_factor()} if {\tt spral\_ssids\_analyse()} was
            called with {\tt check} set to {\tt false} but {\tt ptr[]}
            and/or {\tt row[]} was \texttt{NULL}.
\item{$-$8} Returned by {\tt spral\_ssids\_analyse()}  and
            {\tt spral\_ssids\_analyse\_coord()} if {\tt options.ordering} is
            out-of-range, or {\tt options.ordering}$=${\tt 0} and the user
            has either failed to provide an elimination order or an error has
            been found in the user-supplied elimination order (as supplied in
            {\tt order[]}).
\item{$-$9} Returned by {\tt spral\_ssids\_analyse()} and
            {\tt spral\_ssids\_analyse\_coord()} if {\tt options.ordering}$=${\tt 2},
            but {\tt val[]} was \texttt{NULL}.
\item{$-$10} Returned by {\tt spral\_ssids\_solve()} if there is an error in the size
            of array {\tt x[][]} (that is, {\tt ldx$<$n} or {\tt nrhs$<$1}).
            The user may reset {\tt ldx} and/or {\tt nrhs} and recall
            {\tt spral\_ssids\_solve()}.
\item{$-$11} Returned by {\tt spral\_ssids\_solve()} if {\tt job} is out-of-range.
            The user may reset {\tt job} and recall \\
            {\tt spral\_ssids\_solve()}.
\item{$-$12} Returned by {\tt spral\_ssids\_solve()} and
            {\tt spral\_ssids\_alter()} if the selected combination of
            {\tt options.use\_gpu\_solve} and {\tt options.presolve} are not
            compatible with the requested operation.
\item{$-$13} Returned by {\tt spral\_ssids\_enquire\_posdef()} if
            {\tt posdef}$=${\tt false} on the last call
            to {\tt spral\_ssids\_factor()}.
\item{$-$14} Returned by {\tt spral\_ssids\_enquire\_indef()} if
            {\tt posdef}$=${\tt true} on the last call to
            {\tt spral\_ssids\_factor()}.
\item{$-$15} Returned by {\tt spral\_ssids\_factor()} if {\tt options.scaling$=$3}
            but a matching based ordering was not used during the call to
            {\tt ssids\_analyse()} or {\tt ssids\_analyse\_coord()} (i.e. was
            called with \\
            {\tt options.ordering}$\ne${\tt 2}).
\item{$-$50} Allocation error. If available, the Fortran {\tt stat}
            parameter is returned in {\tt inform.stat}.
\item{$-$51} CUDA error. The CUDA error return value is returned in
            {\tt inform.cuda\_error}.
\item{$-$52} CUBLAS error. The CUBLAS error return value is returned in
            {\tt inform.cublas\_error}.
\end{description}
A positive value of {\tt inform.flag}
is used to warn the user that the input matrix data may be faulty or that
the subroutine cannot guarantee the solution obtained.
Possible values are:
\begin{description}
\item{$+$1} Returned by {\tt spral\_ssids\_analyse()}
and {\tt spral\_ssids\_analyse\_coord()} if out-of-range variable
indices found.
Any such entries are ignored  and the computation continues.
{\tt inform.matrix\_outrange} is set to the number of such entries.

\item{$+$2} Returned by {\tt spral\_ssids\_analyse()} and {\tt spral\_ssids\_analyse\_coord()}
if duplicated indices found. Duplicates are recorded and the corresponding
entries are summed. {\tt inform.matrix\_dup} is set to the number of such entries.

\item{$+$3} Returned by {\tt spral\_ssids\_analyse()} and {\tt spral\_ssids\_analyse\_coord()} if both
out-of-range and duplicated variable indices found.

\item{$+$4} Returned by {\tt spral\_ssids\_analyse()}  and {\tt spral\_ssids\_analyse\_coord()}
if  one and more diagonal entries
of $A$ is missing.

\item{$+$5} Returned by {\tt spral\_ssids\_analyse()}  and {\tt spral\_ssids\_analyse\_coord()}
if  one and more diagonal entries
of $A$ is missing  and
out-of-range and/or duplicated variable indices have been found.

\item{$+$6} Returned by {\tt spral\_ssids\_analyse()}  and {\tt spral\_ssids\_analyse\_coord()} if
$A$ is found be (structurally) singular. This will overwrite any of the above warnings.

\item{$+$7} Returned by {\tt spral\_ssids\_factor()} if {\tt options.action} is set
to {\tt true} and the matrix is found to be (structurally or numerically)
singular.

\item{$+$8} Returned by {\tt spral\_ssids\_factor()} if
{\tt options.ordering}$=${\tt 2} (i.e.
a matching-based ordering was used) but the associated scaling was not (i.e. {\tt options.scaling}$\ne$
{\tt 3}).

\end{description}

\section{Method} \label{ssids: method}

\subsection*{\texttt{spral\_ssids\_analyse()} and \texttt{spral\_ssids\_analyse\_coord()}}
If {\tt check} is set to {\tt true} on the call to {\tt spral\_ssids\_analyse()}
or if {\tt spral\_ssids\_analyse\_coord()} is called, 
the user-supplied matrix data is checked for errors. The cleaned integer matrix data
(duplicates are summed and out-of-range indices discarded) is stored in
{\tt akeep}.
The use of checking is optional on a call to {\tt spral\_ssids\_analyse()} as it
incurs both time and memory overheads. However, it is recommended
since the behaviour of the other routines in the package
is unpredictable if duplicates and/or out-of-range variable indices are entered.

If the user has supplied an elimination order it is checked for errors. Otherwise,
an elimination order is generated by the package. 
The elimination order is used to construct an assembly tree.
On exit from {\tt spral\_ssids\_analyse()} (and {\tt spral\_ssids\_analyse\_coord()}), 
{\tt order[]} is set so that {\tt order[i]} holds the position
of variable $i$ in the elimination order. If an ordering was supplied by the user, this
order may differ, but will be equivalent in terms of fill-in.

%If a matching-based elimination order is requested and {\tt scale[]} is not \texttt{NULL}, on exit,
%{\tt scale[]} contains scaling factors that are computed as a bi-product of
%the ordering algorithm and these may be passed unchanged to {\tt spral\_ssids\_factor()}.

\subsection*{\texttt{spral\_ssids\_factor()}}
{\tt spral\_ssids\_factor()} optionally computes a scaling and then performs the numerical factorization.
 The user must specify whether or not the matrix is
positive definite. If {\tt posdef} is set to {\tt true}, no pivoting
is performed and the computation will terminate with an error if a
non-positive pivot is encountered.

The factorization uses the assembly tree that was set up by the analyse phase.
At each  node,  entries from $A$ and, if it is not a leaf node,
the generated elements and any delayed pivots from its child nodes
must be assembled. Separate kernels handle each of these.


The kernel that performs the assembly from the child nodes considers one parent-child
assembly at a time. Each  generated element from a child is divided into a number of
tiles, and a thread block launched to assemble each tile into a 
dense submatrix using a simple
mapping array to determine the destination row and column of each entry.
Bit-compatibility is achieved by ensuring the child entries are
always assembled in the same order. 

A dense partial factorization of the fully summed columns is then performed. The
fully summed columns are split into a number
of tiles that are each handled by an associated  block. Factorization
proceeds one column of tiles at a time. The pivoting condition is chosen to
ensure that all entries of $L$ have absolute value less than $\texttt{u}^{-1}$.
This limits the growth of the entries of the $D$ factor and ensures that any
solves will be backwards stable. The details are described in [1].

If a pivot candidate does not pass the pivot tests at a node, it is delayed
to its parent node, where further elimination operations may make it acceptable.
Delaying pivots leads to additional fill-in and floating-point
operations beyond that predicted by {\tt spral\_ssids\_analyse()}  (and {\tt spral\_ssids\_analyse\_coord()}), 
and may result in additional memory allocations being required.
The number of delayed pivots can often be reduced by using appropriate scaling.

At each non-root node, the majority of the floating-point operations  involve the formation
of the generated element. 
This is handled by a single dedicated kernel; again, see [1] for details.

At the end of the factorization, data structures for use in future calls to
\texttt{spral\_ssids\_solve()} are prepared. If \texttt{options.presolve=1}, the
block of $L$ corresponding to the eliminated variables is explicitly inverted
to accelerate future calls to \texttt{spral\_ssids\_solve()} at the cost of making
\texttt{spral\_ssids\_factor()} slower.

\subsection*{\texttt{spral\_ssids\_solve()}}
If \texttt{options.use\_gpu\_solve}$=$\texttt{false}, data is moved to the
CPU if required and the BLAS calls are used to perform a solve using the
assembly tree and factors generated on previous calls.

Otherwise, the solve is conducted on the GPU in a similar fashion. If
\texttt{options.presolve=0}, custom GPU implementations of \texttt{\_trsv()}
and \texttt{\_gemv()} are used to handle multiple independent operations. If
multiple right-hand sides are to be solved for, the single right-hand side solve is looped over. If \texttt{options.presolve=1}, \texttt{\_trsv()} can be
replaced by the much more parallel (and hence faster) \texttt{\_gemv()}. In
this case multiple right-hand sides are handled at the same time.

\subsection*{References}
[1] J.D. Hogg, E. Ovtchinnikov and J.A. Scott. (2014).
A sparse symmetric indefinite direct solver for GPU architectures.
RAL Technical Report. RAL-P-2014-0xx, to appear.



%%%%%%%%%%%%%%%%%%%%%% EXAMPLE %%%%%%%%%%%%%%%%%%%%%%%%

\section{Example} \label{ssids: examples}

Suppose we wish to factorize the matrix
\[ A = \left(
\begin{array}{ccccc}
2. & 1. \\
1. & 4. & 1. & & 1. \\
 & 1. & 3. & 2. \\
& & 2. & 0. &  \\
& 1. & & & 2.
\end{array}
\right)
\]
and then solve for the right-hand side
\[ B = \left(
\begin{array}{c}
4. \\
17. \\
19. \\
6. \\
12.
\end{array}
\right).
\]
The following code may be used.
\verbatiminput{examples/C/ssids.c}
This produces the following output:
\begin{verbatim}
 Warning from ssids_analyse. Warning flag =   4
 one or more diagonal entries is missing
The computed solution is:
   1.0000000000e+00   2.0000000000e+00   3.0000000000e+00   4.0000000000e+00   5.0000000000e+00
Pivot order:     3     4     1     0     2
\end{verbatim}

\begin{funders}
   \funder{epsrc}{Funded by EPSRC grant EP/J010553/1}
\end{funders}
